\chapter{Examples}
\label{anexample}

%%%%%%%%%%%%%%%%%%%%%%%%%%%%%%%%%%%%%%%%%%%%%%%%%%%%%%%%%%%%%%%%%%%%%%%%%%%%%%%%%%%%%%%%%%%%%%%%
\section{Fractals}\indexmain{example}
\label{fractals}

The \GrG\ package ships with samples for fractal generation.
We will construct the \indexed{Sierpinski triangle} and the \indexed{Koch snowflake}.
They are created by consecutive rule applications, starting at initial host graphs.

\begin{center}
\includegraphics[width=4cm]{fig/startsir}\quad\quad\includegraphics[width=4cm]{fig/startkoch}
\end{center}

First of all, we have to compile the model and rule set files.
So execute in \GrG's \texttt{examples} directory
\begin{verbatim}
GrGen.exe Sierpinski\Sierpinski.grg
GrGen.exe Snowflake\Snowflake.grg
\end{verbatim}
or
\begin{verbatim}
mono GrGen.exe Sierpinski/Sierpinski.grg
mono GrGen.exe Snowflake/Snowflake.grg
\end{verbatim}
respectively (assuming grgen is contained in your search path, otherwise you have to execute it from the \texttt{bin} directory, adapting the paths to the specification files). 
We can increase the number of iterations to get even more beautiful graphs by editing the file \texttt{Sierpinski.grs} or \texttt{Snowflake.grs}, respectively.
Just follow the comments.
But be careful: the running time increases exponentially with the number of iterations, also \yComp's \indexed{layout algorithm} might need some time and attempts to lay it out nicely.

We execute the Sierpinski script by
\begin{verbatim}
GrShell.exe Sierpinski\Sierpinski.grs
\end{verbatim}
or
\begin{verbatim}
mono GrShell.exe Sierpinski/Sierpinski.grs
\end{verbatim}
respectively. 

Because both of the scripts are using the debug mode, we complete execution by typing \texttt{r}(un).
See Section~\ref{grsthings} for further information.
The resulting graphs should look like Figures~\ref{figsierp} and~\ref{figsnowflake}.
\begin{figure}[htbp]
  \centering
  \includegraphics[width=\textwidth]{fig/sierpinski}
  \caption{Sierpinski triangle}
  \label{figsierp}
\end{figure}
\begin{figure}[htbp]
  \centering
  \includegraphics[width=\textwidth]{fig/snowflake}
  \caption{Koch snowflake}
  \label{figsnowflake}
\end{figure}
\vfill
\pagebreak


%%%%%%%%%%%%%%%%%%%%%%%%%%%%%%%%%%%%%%%%%%%%%%%%%%%%%%%%%%%%%%%%%%%%%%%%%%%%%%%%%%%%%%%%%%%%%%%%
\section{Busy Beaver}\indexmain{example}
We want \GrG\ to work as hard as a \indexed{busy beaver}~\cite{Kro:07,Dew:84}.
A busy beaver is a Turing machine, in our case with five states plus a ``halt''-state; it writes 1,471 bars onto the tape and terminates~\cite{MB:00}. 
We use the graph model and the rewrite rules to define a general Turing machine.
Our approach is basically to draw the machine as a graph.
The busy beaver logic is implemented by rule applications in \GrShell.
Besides giving an example, this shows that \GrG\ is \indexed{Turing complete}.

%-----------------------------------------------------------------------------------------------
\subsection{Graph Model}
So first of all, we design a Turing machine as graph model. 
The tape will be a chain of \texttt{TapePosition} nodes connected by right edges. 
A cell value is modeled by a reflexive \texttt{value} edge, attached to a \texttt{TapePosition} node.
The leftmost and the rightmost cells (\texttt{TapePosition}) do not have an incoming and outgoing edge, respectively.
Therefore, we have the node constraint $[0:1]$.

\begin{grgen}[firstnumber=last]
node class TapePosition;
edge class right
  connect TapePosition[0:1] --> TapePosition[0:1];

edge class value
  connect TapePosition[1] --> TapePosition[1];
edge class zero  extends value;
edge class one   extends value;
edge class empty extends value;
\end{grgen}

Furthermore, we need states and transitions.
The machine's current configuration is modeled with a \texttt{RWHead} edge pointing to a \texttt{TapePosition} node.
\texttt{State} nodes are connected with \texttt{WriteValue} nodes via \texttt{value} edges, a \texttt{moveLeft}/\texttt{moveRight}/\texttt{dontMove} edge leads from a \texttt{WriteValue} node to the next state (cf.~the picture on page \pageref{fig:bbstart}).

\begin{grgen}[firstnumber=last]
node class State;

edge class RWHead;

node class WriteValue;
node class WriteZero extends WriteValue;
node class WriteOne extends WriteValue;
node class WriteEmpty extends WriteValue;

edge class moveLeft;
edge class moveRight;
edge class dontMove;
\end{grgen}

%-----------------------------------------------------------------------------------------------
\subsection{Rule Set}
Now the rule set: We begin the rule set file \texttt{Turing.grg} with

\begin{grgen}[firstnumber=1]
#using "TuringModel.gm"

\end{grgen}
We need rewrite rules for the following steps of the Turing machine:
\begin{enumerate}
  \item Read the value of the current tape cell and select an outgoing edge of the current state.
  \item Write a new value into the current cell, according to the sub type of the \texttt{WriteValue} node.
  \item Move the read-write-head along the tape and select a new state as current state.
\end{enumerate}
As you can see a transition of the Turing machine is split into two graph rewrite steps:
Writing the new value onto the tape, and performing the state transition.
We need eleven rules: Three rules for each step (for ``zero'', ``one'', and ``empty'') and two rules for extending the tape to the left and the right, respectively.
\begin{grgen}[firstnumber=last]
rule readZeroRule {
	s:State -h:RWHead-> tp:TapePosition -:zero-> tp;
	s -:zero-> wv:WriteValue;
	modify {
		delete(h);
		wv -:RWHead-> tp;
	}
}
\end{grgen}

We take the current state \texttt{s} and the current cell \texttt{tp} which is implicitly given by the unique \texttt{RWHead} edge and check whether the cell value is zero.
Furthermore, we check if the state has a transition for zero. 
The replacement part deletes the \texttt{RWHead} edge between \texttt{s} and \texttt{tp} and adds it between \texttt{wv} and \texttt{tp}. 
The remaining rules are analogous:

\begin{grgen}[firstnumber=last]
rule readOneRule {
	s:State -h:RWHead-> tp:TapePosition -:one-> tp;
	s -:one-> wv:WriteValue;
	modify {
		delete(h);
		wv -:RWHead-> tp;
	}
}

rule readEmptyRule {
	s:State -h:RWHead-> tp:TapePosition -:empty-> tp;
	s -:empty-> wv:WriteValue;
	modify {
		delete(h);
		wv -:RWHead-> tp;
	}
}

rule writeZeroRule {
	wv:WriteZero -rw:RWHead-> tp:TapePosition -:value-> tp;
	replace {
		wv -rw-> tp -:zero-> tp;
	}
}

rule writeOneRule {
	wv:WriteOne -rw:RWHead-> tp:TapePosition -:value-> tp;
	replace {
		wv -rw-> tp -:one-> tp;
	}
}

rule writeEmptyRule {
	wv:WriteEmpty -rw:RWHead-> tp:TapePosition -:value-> tp;
	replace {
		wv -rw-> tp -:empty-> tp;
	}
}

rule moveLeftRule {
	wv:WriteValue -:moveLeft-> s:State;
	wv -h:RWHead-> tp:TapePosition <-r:right- ltp:TapePosition;
	modify {
		delete(h);
		s -:RWHead-> ltp;
	}
}

rule moveRightRule {
	wv:WriteValue -:moveRight-> s:State;
	wv -h:RWHead-> tp:TapePosition -r:right-> rtp:TapePosition;
	modify {
		delete(h);
		s -:RWHead-> rtp;
	}
}

rule dontMoveRule {
	wv:WriteValue -:dontMove-> s:State;
	wv -h:RWHead-> tp:TapePosition;
	modify {
		delete(h);
		s -:RWHead-> tp;
	}
}

rule ensureMoveLeftValidRule {
	wv:WriteValue -:moveLeft-> :State;
	wv -:RWHead-> tp:TapePosition;
	negative {
		tp <-:right-;
	}
	modify {
		tp <-:right- ltp:TapePosition -:empty-> ltp;
	}
}

rule ensureMoveRightValidRule {
	wv:WriteValue -:moveRight-> :State;
	wv -:RWHead-> tp:TapePosition;
	negative {
		tp -:right->;
	}
	modify {
		tp -:right-> rtp:TapePosition -:empty-> rtp;
	}
}
\end{grgen}

Have a look at the negative conditions within the \texttt{ensureMove\dots} rules.
They ensure that the current cell is indeed at the end of the tape: An edge to a right/left neighboring cell must not exist. 
Now don't forget to compile your model and the rule set with \texttt{GrGen.exe} (see Section~\ref{fractals}).

%-----------------------------------------------------------------------------------------------
\subsection{Rule Execution with \GrShell}

Finally we construct the busy beaver and let it work with \GrShell. 
The following script starts with building the Turing machine that is modeling the six states with their transitions in our Turing machine model:

\begin{grshell}[firstnumber=1]
select backend "../bin/lgspBackend.dll"
new graph "../lib/lgsp-TuringModel.dll" "Busy Beaver"
select actions "../lib/lgsp-TuringActions.dll"

# Initialize tape
new tp:TapePosition($="Startposition")
new tp -:empty-> tp

# States
new sA:State($="A")
new sB:State($="B")
new sC:State($="C")
new sD:State($="D")
new sE:State($="E")
new sH:State($ = "Halt")

new sA -:RWHead-> tp

# Transitions: three lines per state and input symbol for
#   - updating cell value
#   - moving read-write-head
# respectively

new sA_0: WriteOne
new sA -:empty-> sA_0
new sA_0 -:moveLeft-> sB

new sA_1: WriteOne
new sA -:one-> sA_1
new sA_1 -:moveLeft-> sD

new sB_0: WriteOne
new sB -:empty-> sB_0
new sB_0 -:moveRight-> sC

new sB_1: WriteEmpty
new sB -:one-> sB_1
new sB_1 -:moveRight-> sE

new sC_0: WriteEmpty
new sC -:empty-> sC_0
new sC_0 -:moveLeft-> sA

new sC_1: WriteEmpty
new sC -:one-> sC_1
new sC_1 -:moveRight-> sB

new sD_0: WriteOne
new sD -:empty-> sD_0
new sD_0 -:moveLeft->sE

new sD_1: WriteOne
new sD -:one-> sD_1
new sD_1 -:moveLeft-> sH

new sE_0: WriteOne
new sE -:empty-> sE_0
new sE_0 -:moveRight-> sC

new sE_1: WriteOne
new sE -:one-> sE_1
new sE_1 -:moveLeft-> sC
\end{grshell}

\quad\\Our busy beaver looks like this:\label{fig:bbstart}
\begin{center}
  \fbox{\includegraphics[width=\linewidth-2\fboxsep-2\fboxrule]{fig/bbstart}}
\end{center}

We have an initial host graph now. 
The graph rewrite sequence is quite straight forward and generic to the Turing graph model.
Note that for each state the ``\texttt{\dots Empty\dots} | \texttt{\dots One\dots}'' selection is unambiguous.
%\pagebreak %HACK

\begin{grshell}[firstnumber=last]
  exec ((readOneRule | readEmptyRule) & (writeOneRule | writeEmptyRule) & (ensureMoveLeftValidRule | ensureMoveRightValidRule) & (moveLeftRule | moveRightRule))[32]
\end{grshell}

\quad\\We interrupt the machine after 32 iterations and look at the result so far:
\begin{center}
  \fbox{\includegraphics[width=\linewidth-2\fboxsep-2\fboxrule]{fig/bbmiddle}}
\end{center}
In order to improve the performance we generate better \indexed{search plan}s.
This is a crucial step for execution time:
With the initial search plans the beaver runs for 1 minute and 30 seconds.
With improved search plans after the first 32 steps it takes about 8.5 seconds\footnote{On a Pentium 4, 3.2Ghz, with 2GiB RAM.}.

\begin{grshell}[firstnumber=last]
custom graph analyze
custom actions gen_searchplan readOneRule readEmptyRule writeOneRule writeEmptyRule ensureMoveLeftValidRule ensureMoveRightValidRule moveLeftRule moveRightRule
\end{grshell}

Let the beaver run:
\begin{grshell}[firstnumber=last]
  exec ((readOneRule | readEmptyRule) & (writeOneRule | writeEmptyRule) & (ensureMoveLeftValidRule | ensureMoveRightValidRule) & (moveLeftRule | moveRightRule))*
\end{grshell}

You can see the difference in between the search plans causing that improvement in execution time by utilizing the \texttt{explain} command.

\begin{grshell}
custom actions explain moveRightRule
\end{grshell}

We'll take a look at the search plan of the \texttt{moveRightRule}, first the initial version:

\begin{verbatim}
static search plans
moveRightRule:
    lookup -r:right-> in graph
    from <-r- get source tp:TapePosition
    from -r-> get target rtp:TapePosition
    from tp incoming <-h:rwHead-
    from <-h- get source wv:WriteValue
    from wv outgoing -_edge0:moveRight->
    from -_edge0-> get target s:State
\end{verbatim}

After graph analysis and search plan regeneration it looks like this:

\begin{verbatim}
moveRightRule:
    lookup -h:rwHead-> in graph
    from <-h- get source wv:WriteValue
    from -h-> get target tp:TapePosition
    from wv outgoing -_edge0:moveRight->
    from -_edge0-> get target s:State
    from tp outgoing -r:right->
    from -r-> get target rtp:TapePosition
\end{verbatim}

The crucial difference is the changed lookup, the rule begins matching at the single \texttt{rmHead} edge instead of one of the many \texttt{right} edges on the tape.
