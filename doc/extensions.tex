\chapter{Extensions}\indexmain{extensions}\label{chapextensions}

This chapter explains how to customize \GrG\ with external code, or how to tightly integrate \GrG\ with external code.
It lists the ways how you can interact with the external world outside of \GrG.
The primary means available are: external attribute types and their methods, external functions and procedures, external match filters, and external sequences; the secondary helpers available are annotations, command line parameters, and external shell commands.

You typically want to use them for integrating functionality outside of graph representation processing, i.e. subtasks \GrG\ was \emph{not} built for.
You maybe want to use them for performance reasons, to implement linked lists or trees more efficiently than with the nodes and edges supplied by \GrG.

The languages of \GrG\ shield you from the details of the runtime and the framework below it; you can combine their constructs freely.
But when you plug into the existing framework with your code, you must obey the expectations of that framework.
When you extend \GrG, you must play according to the rules of the \GrG-components.
That said, there are plenty of possibilities to extend the framework with your own functionality.

%-----------------------------------------------------------------------------------------------
\section{External Attribute Types}\label{sub:extcls}
\begin{rail}
  ExternalClassDeclaration: 'external' 'class' IdentDecl (() | 'extends' (ExternalClassType+',')) \\
	(';' | lbrace ((ExternalFunctionDeclaration | ExternalProcedureDeclaration)*) rbrace);
\end{rail}\ixnterm{ExternalClassDeclaration}\ixkeyw{external}\ixkeyw{class}\ixkeyw{extends}
Registers a new attribute type with \GrG. 
You may declare the base types of the type, and you may specify external function methods or procedure methods.
Attributes cannot be specified. 
The attribute type (and the declared methods) must be implemented externally, see \ref{sub:extclsfctimpl}.
For \GrG~the type is opaque, only the function/procedure methods or external functions/procedures can carry out computations on it. 

You may extend \GrG~with external attribute types if the built-in attribute types (cf. \ref{sec:builtintypes}) are insufficient for your needs (because you need functionality not related to graph-rewriting, or because you want to implement lists or trees more efficiently than with the implementation of the node and edge types built for general-purpose graph rewriting).
The external types can be explicitly casted to \texttt{object} but are not implicitly casted.
(The methods may register own undo items with the transaction manager to realize rollback behaviour for external attributes (cf. \ref{ex:transman})).

%-----------------------------------------------------------------------------------------------
\section{External Function Types}\label{sub:extfct}
\begin{rail}
  ExternalFunctionDeclaration: 'external' 'function' IdentDecl '(' ( () | (Type + ',') ) ')' ':' Type ';';
\end{rail}\ixnterm{ExternalFunctionDeclaration}\ixkeyw{external}\ixkeyw{function}
Registers an \indexed{external function} with \GrG~to be used in attribute computations.
An external function declaration specifies the expected input types and the delivered output type.
The function must be implemented externally, see \ref{sub:extclsfctimpl}.
An external function call (cf. \ref{sec:primexpr}) may receive and return values of the built-in (attribute) types, as well as of the external attribute types.
The real arguments on the call sites are type-checked against the declared signature following the subtyping hierarchy of the built-in -- as well as of the external -- attribute types.
You may extend \GrG~with external functions if the built-in attribute computation capabilities (cf. \ref{sub:expr}) or graph querying capabilities are insufficient for your needs.

%-----------------------------------------------------------------------------------------------
\section{External Procedure Types}\label{sub:extproc}
\begin{rail}
  ExternalProcedureDeclaration: 'external' 'procedure' IdentDecl '(' ( () | (Type + ',') ) ')' \\ ( ':' '(' (Type + ',') ')' )? ';';
\end{rail}\ixnterm{ExternalProcedureDeclaration}\ixkeyw{external}\ixkeyw{procedure}
Registers an \indexed{external procedure} with \GrG~to be used in attribute computations.
An external procedure declaration specifies the expected input types and the delivered output types.
The procedure must be implemented externally, see \ref{sub:extclsfctimpl}.
An external procedure call (cf. \ref{sec:primexpr}) may receive and return values of the built-in (attribute) types as well as of the external attribute types.
The real arguments on the call sites are type-checked against the declared signature following the subtyping hierarchy of the built-in -- as well as of the external -- attribute types.
You may extend \GrG~with external procedures if the built-in attribute computation capabilities (cf. \ref{sub:expr}) or graph manipulation capabilities are insufficient for your needs.

%-----------------------------------------------------------------------------------------------
\section{External Filter Functions}\label{sub:extflt}

\begin{rail}
	FilterFunctionDefinition: 'external' 'filter' FilterIdent '<' RuleIdent '>' (Parameters | ()) ';';
\end{rail}\ixnterm{FilterFunctionDefinition}\ixkeyw{filter}\ixkeyw{external}

The \emph{FilterFunctionDefinition} prepended with an \texttt{external} and deprived of its body (cf. \ref{sub:filters} for more on filters)
registers an \indexed{external filter function} with \GrG, for the rule (or test) specified in angles.
It can then be used from applications of that rule (cf. \emph{FilterCalls} in \ref{sec:ruleapplication}).
A filter function name -- irrespective whether external or internal -- must be globally unique (in contrast to the predefined filters --- each rule may offer one with the same name).
The match filter function must be implemented externally, see \ref{sub:extfltseqimpl}.
You may extend \GrG~with match filters if you need to inspect the matches found for a rule, in order to decide which one to apply (see note \ref{note:inspect}), or if you just need a post-match hook which informs you about the found matches.

\subsubsection*{Match object}

For a rule or subpattern \texttt{r}, a match interface \texttt{IMatch\_r} is generated, extending the generic \texttt{IMatch} interface from \texttt{libGr}.
The basic constituents, i.e. nodes, edges and variables are mapped directly to members of their name and type, containing the graph element matched or the value computed.

For alternatives nested inside the pattern, a common base interface \texttt{IMatch\_r\_altName} is generated, plus for each alternative case an interface \texttt{IMatch\_r\_altName\_altCaseName}.
When you walk the matches tree using the type exact interface (which is recommended), you must type switch on the match object in the alternative variable, to determine the case which was finally matched, and cast to its match type.

For iterated patterns nested inside the pattern, an iterated variable of type \texttt{IMatchesExact} \\
\texttt{<IMatch\_r\_iteratedName>} is created in the pattern match object; the \texttt{IMatchesExact} allows you to iterate over the patterns finally found. 

A subpattern usage is mapped directly to a variable in the match object typed with the match interface of the subpattern.

An independent pattern is mapped directly to a variable in the match object typed with the match interface of the independent pattern.

Negative patterns do not appear in the match objects, for they prevent the matching and thus the building of a match object in the first place.

%-----------------------------------------------------------------------------------------------
\section{External Sequences}\label{sub:extseq}
\begin{rail}
  ExternalSequenceDeclaration: 
    'external' 'sequence' RewriteSequenceSignature ';';
\end{rail}\ixnterm{ExternalSequenceDeclaration}\ixkeyw{external}
Registers an \indexed{external sequence} similar to a defined sequence (cf. \ref{sec:sequencedefinition}), but in contrast to that one, it must or can be implemented outside in C\# code.
An external sequence declaration specifies the expected input types and the delivered output types.
The sequence must be implemented externally, see \ref{sub:extfltseqimpl}.
You may extend \GrG~with external sequences if you want to call into external code,
in order to interface with libraries outside of the domain of graph rewriting, 
or if the \GrG-languages are not well suited for parts of the task at hand.

%-----------------------------------------------------------------------------------------------
\section{External Emitting and Parsing}\label{sub:extemitparse}
\begin{rail}
  EmitParseDeclaration: 'external' 'emit' (|'graph') 'class' ';';
\end{rail}\ixnterm{EmitParseDeclaration}\ixkeyw{external}\ixkeyw{emit}\ixkeyw{class}
Tells \GrG~that GRS importing and exporting is to be extended, as well as emitting and debugger display, so that these can handle external types, or the type \texttt{object}.
This allows for a tight integration of external types with the built-in functionality.
It allows to inspect types not defined in \GrG~in the debugger (incl. yComp),
and it allows to serialize/deserialize graphs with types not defined in \GrG.
See \ref{sub:apiextemitparse} for an explanation on what needs to be implemented in this case.

%-----------------------------------------------------------------------------------------------
\section{External Cloning and Comparison}\label{sub:extcopycompare}
\begin{rail}
  CopyCompareDeclaration: 'external' ('copy' | '=''=' | '<' ) 'class' ';';
\end{rail}\ixnterm{CopyCompareDeclaration}\ixkeyw{external}\ixkeyw{copy}\ixkeyw{class}
Tells \GrG~that graph element copying or attribute type comparisons (with equality esp. being used for graph isomorphy comparison) are to be extended, so that they can handle external types, or the type \texttt{object}.
This allows for a tight integration of external types with the built-in functionality, 
and esp. allows to implement value semantics (objects are normally compared by checking reference identity and cloned by copying the reference).
See \ref{sub:apiextcopycompare} for an explanation on what needs to be implemented in this case.

%-----------------------------------------------------------------------------------------------
\section{Shell Commands}

\begin{rail}
  '!' CommandLine
\end{rail}\indexmain{\texttt{"!}}
\emph{CommandLine} is executed as-is by the shell of the operating system.

\begin{rail}
  'external' CommandLine
\end{rail}\indexmain{external}
A method in an extension file is called with the \emph{CommandLine} as string parameter.
This is only possible after \texttt{external emit class} was specified, see \ref{sub:extemitparse} above.
See \ref{sub:apiextemitparse} for an explanation of the method named \texttt{External} that needs to be implemented in this case.

\pagebreak %improve layout

%-----------------------------------------------------------------------------------------------
\section{Shell and Compiler Parameters}

When you want to include external code, you likely have to reference external assemblies.
You can do so by compiler parameters, that are also available as configuration options in the shell.
Debug symbols and source code availability are also of increased importance in this case.

When executing the compiler \texttt{GrGen.exe}, amongst others, the following parameters are admissible:

\noindent \texttt{[mono] GrGen.exe } \texttt{[-keep [<dest-dir>]] [-debug]} \texttt{[-r <assembly-path>]}

The assembly \emph{assembly-path} is linked as reference to the compilation result with the \texttt{-r} parameter.
The \texttt{-keep} parameter causes the generated C\# source files to be kept. If \emph{dest-dir} is omitted, a subdirectory \texttt{tmpgrgen$n$}\footnote{$n$ is an increasing number.} within the current directory will be created. 
When \texttt{-debug} is supplied, the assemblies are compiled with debug information (also some validity checking code is added).

These compiler parameters can be configured in the GrShell, too:
\begin{rail}
  'new' 'add' 'reference' Filename
\end{rail}\ixkeyw{new}\ixkeyw{add}\ixkeyw{reference}
Configures a reference to an external assembly \emph{Filename} to be linked into the generated assemblies, maps to the \texttt{-r} option of \texttt{grgen.exe} (cf. \ref{grgenoptions}).

\begin{rail}
  'new' 'set' 'keepdebug' ('on'|'off')
\end{rail}\ixkeyw{new}\ixkeyw{set}\ixkeyw{keepdebug}\ixkeyw{on}\ixkeyw{off}
Configures the compilation of the generated assemblies to keep the generated files and to add debug symbols.
Maps to the \texttt{-keep} and the \texttt{-debug} options of \texttt{grgen.exe}.


%-----------------------------------------------------------------------------------------------
\section{Annotations}\indexmain{annotation}
\label{annotations}

Identifier \indexed{definition}s can be annotated by \indexedsee{pragma}{annotation}s. 
Annotations are key-value pairs.
\begin{rail}
  IdentDecl: Ident (() | '[' (Ident '=' Constant + ',') ']');
\end{rail}\ixnterm{IdentDecl}
You can use any key-value pairs between the brackets.
For \GrG\ only the identifiers \indexed{prio}, \indexed{maybeDeleted}, \indexed{containment}, \indexed{parallelize}, and \indexed{validityCheck} have an effect, cf. Table \ref{tabannotations}.
But you may use the annotations to transmit information from the specification files to API level where they can be queried.

\pagebreak %improve layout

\begin{table}[htbp]
\begin{tabularx}{\linewidth}{|lllX|} \hline
  \textbf{Key} & \textbf{Value Type} & \textbf{Applies to} & \textbf{Meaning} \\ \hline
  \texttt{prio} & int & node, edge & Changes the ranking of a graph element for \indexed{search plan}s.
    The default is \texttt{prio}=1000.
    Graph elements with high values are likely to appear prior to graph elements with low values in search plans.\\
\hline
  \texttt{maybeDeleted} & boolean & node, edge & Prevents a compiler error when a graph element gets accessed that may be matched homomorphically with a graph element that gets deleted.\\
\hline
  \texttt{containment} & boolean & edge type & Used for XMI export; typically defined by the ecore(/XMI) import.
    Declares an edge type to be an edge type defining a containment relation, which causes XML element containment in the exported XMI.
    Default is \texttt{containment}=false.\\
\hline
  \texttt{parallelize} & int & rule, test & Causes parallelization of the pattern matcher of the action; the value specifies the number of threads to use.\\
\hline
  \texttt{validityCheck} & boolean & node, edge & In case of false, skips the contained-in-graph checks for that node/edge, which are executed in case the debug compiler flag (or shell option) is supplied, before a match is rewritten, esp. in between matches of an all-bracketed rule execution.\\
  & & rule & In case of false, skips the contained-in-graph checks for all nodes/edges in the pattern(s) of the rule.\\
\hline
\end{tabularx}
\caption{Annotations}
\label{tabannotations}
\end{table}

\begin{example}
We search the pattern \texttt{v:NodeTypeA -e:EdgeType-> w:NodeTypeB}.
We have a host graph with about 100 nodes of \texttt{NodeTypeA}, 1,000 nodes of \texttt{NodeTypeB} and 10,000 edges of \texttt{EdgeType}.
Furthermore, we know that between each pair of \texttt{NodeTypeA} and \texttt{NodeTypeB} there exists at most one edge of \texttt{EdgeType}.
\GrG\ can use this information to improve the initial search plan if we adjust the pattern with \texttt{v[prio=10000]:NodeTypeA -e[prio=5000]:EdgeType-> w:NodeTypeB}.
\end{example}

