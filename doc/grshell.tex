\chapter{GrShell Language}\indexmain{GrShell}
\label{chapgrshell}
\GrShell\ is a \indexedsee{shell}{GrShell} application built on top of \LibGr\indexmain{libGr} that offers an execution environment for your generated graph transformations.
It belongs to \GrG's standard equipment.
\GrShell\ is capable of creating, manipulating, and dumping graphs as well as performing and esp. debugging graph rewriting.
The \GrShell\ provides a line oriented scripting language.
\GrShell\ scripts are structured by simple statements separated by line breaks.

%rewrite stuff to be command based instead of splitting commands over several sections?


%%%%%%%%%%%%%%%%%%%%%%%%%%%%%%%%%%%%%%%%%%%%%%%%%%%%%%%%%%%%%%%%%%%%%%%%%%%%%%%%%%%%%%%%%%%%%%%%
\section{Building Blocks}

\GrShell\ is \indexed{case sensitive}.
A line may be empty, may contain a shell command, or may contain a comment.
A \indexed{comment} starts with a \indexed{\texttt{\#}} and is terminated by end-of-line
 or end-of-file or \texttt{\S}.
The following items are required for representing text, numbers, and rule parameters.\\
\\
\emph{Text}\\
May be one of the following:
\begin{itemize}
  \item A non-empty character sequence consisting of letters, digits, and underscores. The first character must not be a digit.
  \item Arbitrary text enclosed by double quotes (\texttt{""}).
  \item Arbitrary text enclosed by single quotes (\texttt{''}).
\end{itemize}
Due to the chosen parser generator shell keywords are not allowed for type names, attribute values and other entities (even if they are legal in the rule language). If this hits you, you can enclose the identifier by single or double quotes, i.e. Text can be used everywhere an identifier is required.

\mbox{ }\\
\emph{Number}\\
Is an \texttt{int} or \texttt{float} constant in decimal notation (see also Section~\ref{sec:builtintypes}).

\begin{rail}
 Parameters : Text + ',' ;
 SpacedParameters: Text + ;
\end{rail}\ixnterm{Parameters}\ixnterm{SpacedParameters}

In order to describe the commands more precisely, the following (semantic) specializations of \emph{Text} are defined:
\begin{description}
  \item[Filename]A fully qualified file name without spaces (e.g.\ \texttt{/Users/Bob/amazing\textunderscore file.txt}) or a single quoted or double quoted fully qualified file name that may contain spaces (\texttt{"/Users/Bob/amazing file.txt"}).
  \item[Variable] Identifier of a (graph global) variable that contains a graph element or a value. \indexmainsee{GrShell variable}{graph global variable} A double colon prefix as required in the sequences may be given, but as the shell only knows graph global variables, it is optional.
  \item[NodeType, EdgeType] Identifier of a node type resp.\ edge type defined in the model of the current graph.
  \item[AttributeName] Identifier of an attribute.
  \item[Graph] Identifies a graph by its name.
  \item[Action] Identifies a rule by its name.
\end{description}
\makeatletter
\begin{rail}
  GraphElement: Text | ('@' '(' Text ')')
\end{rail}\indexmain{\texttt{"@}}\ixnterm{GraphElement}
\makeatother

The elements of a graph (nodes and edges) can be accessed both by their (graph global) \indexed{variable}\indexmain{graph global variable} identifier and by their \newterm{persistent name} specified through a constructor (see Section~\ref{mani}).
The specializations \emph{Node} and \emph{Edge} of \emph{GraphElement} require the corresponding graph element to be a node or an edge respectively.

\begin{example}
\label{persistentex}
We insert a node, \indexed{anonymous}ly and with a \indexed{constructor} (see also Section~\ref{mani}):
\begin{grshell}
> new graph "../lib/lgsp-TuringModel.dll" G
New graph "G" of model "Turing" created.

# insert an anonymous node...
# it will get a persistent pseudo name
> new :State
New node "$0" of type "State" has been created.
> delete node @("$0")

# and now with constructor
> new v:State($=start)
new node "start" of type "State" has been created.
# Now we have a node named "start" and a variable v assigned to "start"
\end{grshell}
\end{example}

\begin{note}
Persistent names will be saved (\texttt{save graph\dots}, see Section~\ref{outputcmds}) and exported,
and, if you visualize a graph (\texttt{dump graph\dots}, see Section~\ref{outputcmds}),
graph elements will be \indexed{label}ed with their persistent names.
Persistent names have to be unique for a graph (the graph they belong to).
\end{note}


%%%%%%%%%%%%%%%%%%%%%%%%%%%%%%%%%%%%%%%%%%%%%%%%%%%%%%%%%%%%%%%%%%%%%%%%%%%%%%%%%%%%%%%%%%%%%%%%
\section{Variables}

\begin{rail}
  Variable '=' ( GraphElement | Variable | Literal )
\end{rail}
Assigns the variable or persistent name \emph{GraphElement} or literal to \emph{Variable}.
If \emph{Variable} has not been defined yet, it will be defined implicitly.
As usual for scripting languages, variables have neither static types nor declarations.
The variables known to \GrShell\ are the graph global variables (see \ref{cha:xgrs} for the distinction between graph global and sequence local variables).

\begin{rail}
'show' 'var' Variable
\end{rail}\ixkeyw{show}\ixkeyw{var}
Prints the content of the specified variable.

\begin{rail}
  'askfor';
  Variable '=' 'askfor' Type
\end{rail}\ixkeyw{askfor}
The \texttt{askfor} command just waits until the user presses enter.
The \texttt{askfor} assignment interactively asks the user for a value of the specified type.
The entered value is type checked against the expected type, and assigned to the given variable in case it matches.
If the type is a value type, the user is prompted to enter a value literal with the keyboard.
If the type is a graph element type, the user is prompted to enter the graph element by double clicking in yComp.
Note that in this case the debug mode must have been enabled before.
(The command is equivalent to \verb#debug exec Variable=$%(Type)#.)

\begin{example}
\begin{grshelllet}
x = askfor int
\end{grshelllet}
asks the user to enter an integer value; pressing 4 then 2 then enter will do fine.
\begin{grshelllet}
x = askfor Node
\end{grshelllet}
asks the user to select a graph element in yComp; double clicking any node will do fine.
\end{example}


%%%%%%%%%%%%%%%%%%%%%%%%%%%%%%%%%%%%%%%%%%%%%%%%%%%%%%%%%%%%%%%%%%%%%%%%%%%%%%%%%%%%%%%%%%%%%%%%
\section{Common and File System Commands}
\label{commcommands}

Here and in the following sections we describes the \GrShell\ commands\ixnterm{Command}.
Commands are assembled from basic elements.
As stated before commands are terminated by line breaks.
Alternatively commands can be terminated by the \indexed{\texttt{;;}} symbol.
Like an operating system shell, the \GrShell\ allows you to span a single command over $n$ lines by terminating the first $n-1$ lines with a \indexed{backslash}.
\begin{rail}
  Script: ((Command ('<line break>' | ';;'))+) '<end of file>' ;
\end{rail}\ixnterm{Script}

\begin{rail}
  'help' (Command)?
\end{rail}\ixkeyw{help}
Displays an information message describing all the supported commands.
A command \texttt{Command} displayed with \texttt{...} has further help available, which can be displayed with \texttt{help Command}.

\begin{rail}
  'quit' | 'exit'
\end{rail}\ixkeyw{quit}\ixkeyw{exit}
Quits \GrShell. If \GrShell\ is opened in debug mode, a currently active graph viewer (such as \yComp) will be closed as well.

\begin{rail}
  'include' Filename ('.gz')?
\end{rail}\ixkeyw{include}
Executes the \GrShell\ script\indexmain{graph rewrite script} \emph{Filename} (which might be zipped).
A \GrShell\ script is just a plain text file containing \GrShell\ commands.
They are treated as they would be entered interactively, except for parser errors.
If a parser error occurs, execution of the script will stop immediately.

\begin{rail}
  'echo' Text
\end{rail}\ixkeyw{echo}
Prints \emph{Text} onto the \GrShell\ command prompt.

\begin{rail}
  'pwd'
\end{rail}\ixkeyw{pwd}
Prints the path to the current working directory.

\begin{rail}
  'ls'
\end{rail}\ixkeyw{ls}
Lists the directories and files in the current working directory, files relevant to GrGen are printed highlighted.

\begin{rail}
  'cd' Path
\end{rail}\ixkeyw{cd}
Changes the current working directory to the path given.

\begin{rail}
  '!' CommandLine
\end{rail}\indexmain{\texttt{"!}}
\emph{CommandLine}\indexmain{command line} is an arbitrary text, the operating system attempts to execute.
\begin{example}
On a Linux machine you might execute
\begin{grshell}
!sh -c "ls | grep stuff"
\end{grshell}
\end{example}


%%%%%%%%%%%%%%%%%%%%%%%%%%%%%%%%%%%%%%%%%%%%%%%%%%%%%%%%%%%%%%%%%%%%%%%%%%%%%%%%%%%%%%%%%%%%%%%%
\section{Graph Creation}
\label{graphcreationcommands}

The command most shell scripts starts with is graph creation.

\begin{rail}
  'new' (() | 'new') 'graph' Filename Text
\end{rail}\ixkeyw{new}\ixkeyw{graph}
Creates a new graph with the model specified in \emph{Filename}\indexmain{graph model}.
Its name is set to \emph{Text}.
The model file can be either source code (e.g.\ \texttt{turing\textunderscore machineModel.cs}) or a .NET assembly (e.g.\ \texttt{lgsp-turing\textunderscore machineModel.dll}).
It's also possible to specify a rule set file as \emph{Filename} (this is the most common usage).
In this case the necessary assemblies will be created on the fly (as needed).
In case of a double \texttt{new}, the actions and model are created anew for sure, and not as needed according to the file change dates.
Use this if you are working with one of the \texttt{new set} or \texttt{new add} commands from below.

\begin{warning}
You may run into unexpected results because some \texttt{new set} or \texttt{new add} options that you apply and see in the shell file are not the ones actually compiled into the generated code.
This happens easily when you just edit those options, but the actions are not regenerated because the sources did not change.
Use \texttt{new new} in case you are working with the options from below to ensure the actions are regenerated irrespective of the change dates of their sources (at the cost of steady recompilations).
\end{warning}

The following two commands create graph elements, initializing their attributes.
On shell level they are available and mainly used as elementary instructions in creating an initial graph, in exporting and importing a graph, as well as in change recording and replaying.
These are the commands you may find in the GRS export/import files.


\begin{note}
If you need to import data in a format not supported by \GrG, serialize data into the simple format explained here in \ref{graphcreationcommands} with its few \texttt{new} commands and the attribute initialization lists.
It can be written easily.
\end{note}

\begin{rail}
  'new' (() | Text) (() | ':' NodeType (() | Constructor))
\end{rail}\ixkeyw{new}
Creates a new node within the current graph.
Optionally a variable \emph{Text} is assigned to the new node.
If \emph{NodeType} is supplied, the new node will be of type \emph{NodeType} and attributes can be initialized by a constructor.
Otherwise the node will be of the base node class type \emph{Node}.

\begin{note}
The \GrShell\ can reassign \indexed{variable}s.
This is in contrast to the rule language (Chapter~\ref{chaprulelang}), where we mainly use \emph{names}\indexmain{name}\indexmain{expression variable}\indexmainsee{expression variable}{name}
(with exception of var and ref input variables and def entities).
\end{note}

\begin{rail}
  'new' Node (('-' EdgeEntityConstructor '->') | ('<-' EdgeEntityConstructor '-') | ('-' EdgeEntityConstructor '-')) Node ;
EdgeEntityConstructor:
  (()|Text) (() | ':' EdgeType (() | Constructor)) ;
\end{rail}\ixkeyw{new}
Creates a new edge within the current graph between the specified nodes,
directed from the first to the second \emph{Node} in the case of \texttt{-->},
directed from the second to the first \emph{Node} in the case of \texttt{<--},
or undirected in the case of \texttt{--}.
Optionally a variable \emph{Text} is assigned to the new edge.
If \emph{EdgeType} is supplied, the new edge will be of type \emph{EdgeType} and attributes can be initialized by a constructor.
Otherwise the edge will be of the base edge class type \texttt{Edge} for \texttt{-->} or \texttt{UEdge} for \texttt{--}.

\begin{rail}
  Constructor : '(' (() | (dollar '=' Text (() | ',' Attributes) | Attributes)) ')';
  Attributes : (AttributeName '=' AttributeValue) + (',');
  AttributeValue :  PrimitiveAttributeValue | SetConstr | MapConstr | ArrayConstr | DequeConstr;
  PrimitiveAttributeValue : EnumLit | Number | FloatingNumber | QuotedText | BoolLit | NullLit ;
  SetConstr: 'set' '<' Type '>' lbrace ( Expression*',' ) rbrace ;
  MapConstr: 'map' '<' Type ',' Type '>' \\ lbrace ( (Expression '->' Expression)*',' ) rbrace ;
  ArrayConstr: 'array' '<' Type '>' '[' ( Expression*',' ) ']' ;
  DequeConstr: 'deque' '<' Type '>' ']' ( Expression*',' ) '[' ;
\end{rail}\indexmain{\texttt{\$}}\ixnterm{Constructor}\ixnterm{Attributes}\ixnterm{AttributeValue}\ixnterm{PrimitiveAttributeValue}\ixnterm{SetConstr}\ixnterm{MapConstr}\ixnterm{ArrayConstr}\ixnterm{DequeConstr}
A \indexed{constructor} is used to initialize a new graph element (see \texttt{new \dots} below).
A comma separated list of \indexed{attribute} declarations is supplied to the constructor.
Available attribute names are specified by the graph model of the current working graph.
All the undeclared attributes will be initialized with \indexed{default value}s, depending on their type
(\texttt{int} $\leftarrow$ \texttt{0}; \texttt{long} $\leftarrow$ \texttt{0L}; \texttt{byte} $\leftarrow$ \texttt{0Y}; \texttt{short} $\leftarrow$ \texttt{0S}; \texttt{boolean} $\leftarrow$ \texttt{false}; \texttt{float} $\leftarrow$ \texttt{0.0f}; \texttt{double} $\leftarrow$ \texttt{0.0}; \texttt{string} $\leftarrow$ \texttt{""}; \texttt{set<T>} $\leftarrow$ \texttt{set<T>\{\}}; \texttt{map<S,T>} $\leftarrow$ \texttt{map<S,T>\{\}}; \texttt{array<T>} $\leftarrow$ \texttt{array<T>[]}; \texttt{deque<T>} $\leftarrow$ \texttt{deque<T>][}; \texttt{enum} $\leftarrow$ unspecified;).\\
The \texttt{\$} is a special attribute name: a unique identifier of the new graph element.
This identifier is also called \newterm{persistent name} (see Example~\ref{persistentex}).
This name can be specified by a constructor only.

\begin{rail}
  'add' 'new' 'graph' Text
\end{rail}\ixkeyw{add}\ixkeyw{new}\ixkeyw{graph}
Creates a new subgraph of the same model as the current graph.
Its name is set to \emph{Text}; its unique name as used in exporting or recording may be a variant of this name in case another subgraph of the same name alreay exists.
After execution this graph is the current subgraph.

\begin{rail}
  'in' Text
\end{rail}\ixkeyw{in}
Switches graph processing to the given subgraph (denoted by its unique name).
This command as well as the command above are supplied for importing grs files containing subgraph attributes resulting from a grs export or recording.
Using them directly is discouraged; you would have to ensure unique names in order to use them, a task that is carried out by the exporter remembering the subgraphs already seen.


%%%%%%%%%%%%%%%%%%%%%%%%%%%%%%%%%%%%%%%%%%%%%%%%%%%%%%%%%%%%%%%%%%%%%%%%%%%%%%%%%%%%%%%%%%%%%%%%
\section{Attribute Assignment and Graph Manipulation}
\label{mani}
Graph manipulation commands alter existing graphs;
they allow to retype and delete graph elements and change attributes.
Creating elements was already introduced in the previous section.
These are tasks which are or at least should be carried out by the rules of the rule language in the first place.
On shell level they are available and mainly used as elementary instructions in creating an initial graph, in exporting and importing a graph, as well as in change recording and replaying.

\begin{rail}
  GraphElement '.' AttributeName '=' AttributeValue ;
\end{rail}
Set the \indexed{attribute} \emph{AttributeName} of the graph element \emph{GraphElement} to the value \emph{AttributeValue} (for the different possible attribute values see above).

\begin{rail}
  GraphElement '.' AttributeName '[' Index ']' '=' AttributeValue ;
\end{rail}
Overwrite the value in the array or deque or map \indexed{attribute} \emph{AttributeName} of the graph element \emph{GraphElement} at the integer position or key value \emph{Index} with the value \emph{AttributeValue}.

\begin{rail}
  GraphElement '.' AttributeName '.' 'add' '(' \\
  	PrimitiveAttributeValue (',' PrimitiveAttributeValue)? ')' ;
\end{rail}
Add the value \emph{PrimitiveAttributeValue} (for the different possible primitive attribute values see above) to the set valued attribute \emph{AttributeName} of the graph element \emph{GraphElement} or add the key-value pair consisting of the two \emph{PrimitiveAttributeValue}s to the map valued attribute \emph{AttributeName} of the graph element \emph{GraphElement}.
Or add the value \emph{PrimitiveAttributeValue} to the end of the array/deque valued attribute \emph{AttributeName} of the graph element \emph{GraphElement} in the one parameter case or insert the \emph{PrimitiveAttributeValue} to the of the array/deque valued attribute \emph{AttributeName} of the graph element \emph{GraphElement} at the index given by the second parameter in the two parameter case.

\begin{rail}
  GraphElement '.' AttributeName '.' 'rem' '(' (PrimitiveAttributeValue)? ')' ;
\end{rail}
Remove the value \emph{PrimitiveAttributeValue} from the set valued attribute \emph{AttributeName} of the graph element \emph{GraphElement} or remove the key \emph{PrimitiveAttributeValue} from the map valued attribute \emph{AttributeName} of the graph element \emph{GraphElement} or remove the index \emph{PrimitiveAttributeValue} from the array valued attribute \emph{AttributeName} of the graph element \emph{GraphElement} or remove the index \emph{PrimitiveAttributeValue} from the deque valued attribute \emph{AttributeName} of the graph element \emph{GraphElement} or remove the end element from the array valued attribute \emph{AttributeName} of the graph element \emph{GraphElement} in the zero parameter case or remove the first element from the deque valued attribute \emph{AttributeName} of the graph element \emph{GraphElement} in the zero parameter case.


\begin{rail}
  'retype' Node '<' Type '>'
\end{rail}\ixkeyw{retype}
Retypes the node \emph{Node} from its current type to the new type \emph{Type}. Attributes common to initial and final type are kept. Incident edges are kept as well. \indexmain{retype}

\begin{rail}
  'retype' ('-' Edge '<' Type '>' '->' | '-' Edge '<' Type '>' '-')
\end{rail}\ixkeyw{retype}
Retypes the edge \emph{Edge} from its current type to the new type \emph{Type}. Attributes common to initial and final type are kept. Incident nodes are kept as well.

\begin{rail}
  'redirect' Edge ('source'|'target') Node
\end{rail}\ixkeyw{redirect}
Redirects the edge \emph{Edge} from the old source or target node to the new source or target \emph{Node} given.

\begin{rail}
  'delete' 'node' Node
\end{rail}\ixkeyw{delete}\ixkeyw{node}
Deletes the node \emph{Node} from the current graph.
Incident edges will be deleted as well.

\begin{rail}
  'delete' 'edge' Edge
\end{rail}\ixkeyw{delete}\ixkeyw{edge}
Deletes the edge \emph{Edge} from the current graph.

\begin{rail}
  'clear' 'graph' (() | Graph)
\end{rail}\ixkeyw{clear}\ixkeyw{graph}
Deletes all graph elements of the current working graph resp.\ the graph \emph{Graph}.


%%%%%%%%%%%%%%%%%%%%%%%%%%%%%%%%%%%%%%%%%%%%%%%%%%%%%%%%%%%%%%%%%%%%%%%%%%%%%%%%%%%%%%%%%%%%%%%%
\section{Graph Input and Output}
\label{outputcmds}

\begin{rail}
  'save' 'graph' Filename
\end{rail}\ixkeyw{save}\ixkeyw{graph}
Dumps\indexmain{dumping graph} the current graph as \GrShell\ script\indexmain{graph rewrite script} into \emph{Filename}.
The created script includes
\begin{itemize}
  \item selecting the backend
  \item creating a new graph with all nodes and edges (including their persistent names)
  \item restoring the (graph global) variables
  \item restoring the visualisation styles
\end{itemize}
but not necessarily using the same commands you typed in during construction.
Such a script can be loaded and executed by the \texttt{include} command (see Section~\ref{commcommands}).

\begin{rail}
  'export' Filename ('.grs' | '.grsi') ('.gz')? \\ ( () | 'nonewgraph') (('skip/type.attr')*)
\end{rail}\ixkeyw{export}\indexmain{export}
Exports an instance graph in GRS (.grs/.grsi) format, which is a reduced \GrShell\ script
(it can get imported and exported on API level without using the \GrShell, see \ref{sub:imexport}).
This is the recommended standard format (it is lightweight, human-readable and editable, and supported by an optimized importer).
The file contains a \texttt{new graph} command, followed by \texttt{new node} commands, followed by \texttt{new edge} commands.
If the \texttt{.gz} suffix is given the graph is saved zipped.
The export is only complete with the model of the graph given in the \texttt{.gm} file.
Exporting fails if the graph model contains attributes of \texttt{object}-type; you may add support for storing them, too, see \ref{sub:extemitparse} for more on this.

When the optional parameter \texttt{nonewgraph} is given, the initial \texttt{new graph} command at file begin is omitted.
Such a file cannot be \texttt{import}ed, but only \texttt{include}d, as it is incomplete.
You have to ensure an empty graph of correct model exists before you can include a file exported this way.
	
The skip parameters allow to exclude attributes from the node/edge attribute initializer lists of the \texttt{new node} or \texttt{new edge} commands.
The parameter \texttt{skip/type.attr} causes omission of attribute \texttt{attr} from graph element type \texttt{type} during graph serialization.
This way you can export an again importable graph if you intend to remove some attributes -- otherwise import would fail due to an unknown attribute getting initialized (in the file exported adhering to the old graph model that is not existing in the new graph model).

The \texttt{save} command from above is for saving a \GrShell\ session including graph global variables and visualization commands,
the goal of the \texttt{export} command is simply persistent storing of graphs;
esp. for applications that use \GrG{ }to get an algorithmic core, but are not built on the \GrShell.

\begin{rail}
  'export' Filename '.gxl' ('.gz')?
\end{rail}\ixkeyw{export}\indexmain{export}
Exports an instance graph and a graph model in GXL format \cite{GXL,GXL2},
which is somewhat of a standard format for graphs of graph rewrite systems,
but suffers from the well-known XML problems -- it is barely human-readable and editable, and bloated.
It is supported by \GrG{} as exchange format for inter-tool operability.
Exporting fails if the graph model contains attributes of container or \texttt{object}-type.
If the \texttt{.gz} suffix is given the graph is saved zipped.

\begin{rail}
  'export' Filename '.xmi' ('.gz')?
\end{rail}\ixkeyw{export}\indexmain{export}
Exports an instance graph in .XMI format.
XMI files as written by the Eclipse Modeling Framework (EMF) are a standard format in the model transformation community (together with ecore files for the model).
It suffers from the XML problems explained above, in addition it can be even characterized as overly complex and baroque, and it requires some metamodel mapping.
It is supported by \GrG{} as exchange format for inter-tool operability.
The metamodel is assumed to stem from a previous import of an ecore file, with its specific way of mapping \texttt{.ecore} to \texttt{.gm}, i.e. with an underscore prefix, a node type prefix for the edge types, and the \verb#[containment=true]# annotation at the edges that express containment, which is needed so that they are written with XML node containment.
If the \texttt{.gz} suffix is given the graph is saved zipped.

%\pagebreak %force better layout

\begin{rail}
  'export' Filename '.grg' ('.gz')?
\end{rail}\ixkeyw{export}\indexmain{export}
Exports an instance graph in GRG format, i.e. as one GrGen rule with an empty pattern and a large modify part.
There is no importer existing, this format is not for normal use as storage format!
If the \texttt{.gz} suffix is given the graph is saved zipped.

\begin{rail}
  'import' Filename ('.grs' | '.grsi' ) ('.gz')? (ModelOverride)?
\end{rail}\ixkeyw{import}
Imports the specified graph instance in GRS (.grs/.grsi) format (the \emph{reduced} \GrShell\ script,
a \texttt{save}d graph can only be imported by \texttt{include} (but an exported graph can be imported by \texttt{include}, too)).
The graph model referenced in the .grs/.grsi must be available as \texttt{.gm}-file.
If a model override of the form \texttt{Filename.gm} is specified, the given model will be used instead of the model referenced in the GRS file.
If a model override of the form \texttt{Filename.grg} is specified, the model(s) of the given rule file will be used instead of the model in the GRS file.
If the \texttt{.gz} suffix is given the graph is expected to be zipped.

\begin{rail}
  'import' Filename '.gxl' ('.gz')? (ModelOverride)?
\end{rail}\ixkeyw{import}
Imports the specified graph instance and model in GXL format.
If a model override of the form \texttt{Filename.gm} is specified, the given model will be used instead of the model in the GXL file.
If a model override of the form \texttt{Filename.grg} is specified(s), the model of the given rule file will be used instead of the model in the GXL file.
The \texttt{.gxl}-graph must be compatible to the \texttt{.gm}-model/\texttt{.grg}-model.
If the \texttt{.gz} suffix is given the graph is expected to be zipped.

\begin{note}\label{shellgxlimport}
Normally you are not only interested in importing a GXL graph (and viewing it), but you want to execute actions on it.
The problem is that the actions are model dependent.
So, in order to apply actions, you must use a model override, which works this way:
\begin{enumerate}
\item \texttt{new graph "YourName.grg"}\\
This creates the model library lgsp-YourNameModel.dll
and the actions library lgsp-YourNameActions.dll
(which depends on the model library generated from the \texttt{"using YourName;"}).
\item \texttt{import InstanceGraphOnly.gxl YourName.gm}\\
This imports the instance graph from the .gxl but uses the model specified
in YourName.gm (it must fit to the model in the .gxl in order to work).
\item \texttt{select actions lgsp-YourNameActions.dll}\\
This loads the actions from the actions library in addition to the already
loaded model and instance graph (cf. \ref{grsthings}).
\item Now you are ready to use the actions.
\end{enumerate}
As of version 3.0beta you can specify a \texttt{.grg} as model override;
basically it does what the given enumeration does.
\end{note}

\begin{rail}
  'import' ((Filename '.ecore')+( )) Filename '.xmi' (Filename '.grg')?
\end{rail}\ixkeyw{import}\label{shellecoreexport}
Imports the specified graph instance in XMI format and the models in ecore format.
They can't be imported directly, as \GrG{ } is not built on EMF.
Instead, during the import process an intermediate \texttt{.gm} is written which is equivalent to the \texttt{.ecore} given -- you may inspect it to see how the content gets mapped.
(The importer maps packages to GrGen packages, classes to GrGen node classes, their attributes to corresponding GrGen attributes, and their references to GrGen edge classes.
Inheritance is transferred one-to-one, and enumerations are mapped to GrGen enums.
Edge type names are prefixed by the names of the node types they originate from to prevent name clashes for references of same name,
and all types are prefixed by an underscore to prevent name clashes with keywords of the rule language.
Edge type declarations are annotated with a \verb#[containment=true]# annotation if they originate from a containment reference.)
After this metamodel transformation the instance graph XMI adhering to the Ecore model thus adhering to the just
generated equivalent GrGen graph model gets imported.
Furthermore, you can specify a \texttt{.grg} containing the rules to apply (using further rule and model files).
Some examples stemming from old GraBaTs/TTC challenges export XMI with emit statements, this is not needed anymore with the built-in XMI export.

\begin{rail}
  'import' 'add' FileSpec
\end{rail}\ixkeyw{import}\ixkeyw{add}
Imports the graph in the specified file and adds it to the current graph
(instead of overwriting the old graph with the new graph).
The \texttt{FileSpec} is of the same format as the file specification in the other import commands.


%%%%%%%%%%%%%%%%%%%%%%%%%%%%%%%%%%%%%%%%%%%%%%%%%%%%%%%%%%%%%%%%%%%%%%%%%%%%%%%%%%%%%%%%%%%%%%%%
\section{Graph Change Recording and Replaying}
\label{recordnreplay}

Graph change recording and replaying is available 
\begin{itemize}
	\item for one for post-problem debugging, you can execute your transformation, and after a problem occured inspect the changes that were leading to it
	\item for the other for ensuring persistence of changes as they happen, in case you are using \GrG\ as an application-embedded in-memory graph-datebase
\end{itemize}

\begin{rail}
  'record' Filename ('.gz')? ('start' | 'stop')?
\end{rail}\ixkeyw{record}\ixkeyw{start}\ixkeyw{stop}\indexmain{record}
The record command starts or stops recording of graph changes to the specified file. If neither start nor stop are given, recording to the specified file is toggled (i.e. started if no recording to the file is underway or stopped if the file is already recorded to).
Recording starts with an export (cf. \ref{outputcmds}) of the instance graph in GRS (.grs/.grsi) format, afterwards the command returns but all changes to the instance graph are recorded to the file until the recording stop command is issued.
Furthermore the values given in the \texttt{record} statements (cf. \ref{recstmt}) from the sequences are written to the recording (this allows you to mark states).
If the \texttt{.gz} suffix is given the recording is saved zipped.
You may start and stop recordings to different files at different times, every file receives the graph changes and records statements occurring during the time of the recording.
Note: As a debugging help a recording does not only contain graph manipulation commands (cf. \ref{mani}) but also comments telling about the rewrites and transaction events which occurred (whose effects were recorded).

\begin{rail}
  'recordflush'
\end{rail}\ixkeyw{recordflush}
Flushes the buffers of the recordings to disk. 
To be called to guarantee persistence if you use \GrG{} as a kind of online database, recording the graph changes while running to a redo log.

\begin{rail}
  'replay' Filename ('.gz')? ('from' Text)? ('to' Text)?
\end{rail}\ixkeyw{replay}\ixkeyw{from}\ixkeyw{to}\indexmain{replay}
The replay command plays a recording back: the graph at the time the recording was started is recreated, then the changes which occurred are carried out again, so you end up with the graph at the time the recording was stopped. Instead of replaying the entire GRS file you may restrict replaying to parts of the file by giving the line to start at and/or the line to stop at. Lines are specified by their textual content which is searched in the file.
If a \emph{from} line is given, all lines from file begin on including this line are skipped, then replay starts. If a \emph{to} line is given, only the lines from the starting point on, until-excluding this one are executed (i.e. all lines from-including this one until file end are skipped).
Normally you reference with \texttt{from} and \texttt{to} comment lines you write with the \texttt{record} statement (cf. \ref{recstmt}) in the sequences, marking relevant states during a transformation process.
An example for record and replay is given in \texttt{tests/recordreplay}.


%%%%%%%%%%%%%%%%%%%%%%%%%%%%%%%%%%%%%%%%%%%%%%%%%%%%%%%%%%%%%%%%%%%%%%%%%%%%%%%%%%%%%%%%%%%%%%%%
\section{Shell and Environment Configuration}

\begin{rail}
'silence' ('on'|'off')
\end{rail}\ixkeyw{silence}\ixkeyw{on}\ixkeyw{off}
Switches the new node / edge created / deleted messages on(default) or off.
Switching them off allows for much faster execution of scripts containing a lot of creation commands.

\begin{rail}
'silence' 'exec' ('on'|'off')
\end{rail}\ixkeyw{silence}\ixkeyw{exec}\ixkeyw{on}\ixkeyw{off}
During non-debug sequence execution every second match statistics are printed to the console;
they allow to assess the progress of long-running transformations.
With this command they can be disabled (or enabled again).
Switching them off may be of interest if own debug messages printed via emit from the sequences (or rules) should not be disturbed.

\begin{rail}
'randomseed' (Number | 'time')
\end{rail}\ixkeyw{randomseed}\ixkeyw{time}
Sets the random seed to the given number for reproducible results when using the \$-operator-prefix or the random-match-selector, whereas time sets the random seed to the current time in ms.

\begin{rail}
'redirect' 'emit' Filename
\end{rail}\ixkeyw{redirect}\ixkeyw{emit}
Redirects the output of the emit-statements (but not the emitdebug-statements) in the rules from stdout to the given file.

\begin{rail}
'redirect' 'emit' '-'
\end{rail}\ixkeyw{redirect}\ixkeyw{emit}
Redirects the output of the emit-statements in the rules to stdout (again).


%%%%%%%%%%%%%%%%%%%%%%%%%%%%%%%%%%%%%%%%%%%%%%%%%%%%%%%%%%%%%%%%%%%%%%%%%%%%%%%%%%%%%%%%%%%%%%%%
\section{Compilation Configuration}\label{sec:compilerconfigshell}

\begin{rail}
  'new' 'add' 'reference' Filename
\end{rail}\ixkeyw{new}\ixkeyw{add}\ixkeyw{reference}
Configures a reference to an external assembly \emph{Filename} to be linked into the generated assemblies, maps to the \texttt{-r} option of \texttt{grgen.exe} (cf. \ref{grgenoptions}).

\begin{rail}
  'new' 'set' ('keepdebug'|'lazynic'|'noinline'|'profile') ('on'|'off')
\end{rail}\ixkeyw{new}\ixkeyw{set}\ixkeyw{keepdebug}\ixkeyw{lazynic}\ixkeyw{noinline}\ixkeyw{profile}\ixkeyw{on}\ixkeyw{off}
Configures the compilation of the generated assemblies to keep the generated files and to add debug symbols (includes emitting of some validity checking code),
or configures the generation of the matchers.
The latter to either execute negatives, independents, and conditions only at the end of matching (normally asap),
or to never inline subpatterns,
or to include profiling information.
Those flags are mapped to the \texttt{-keep} and the \texttt{-debug} options, or to the \texttt{-lazynic}, \texttt{-noinline}, or  \texttt{-profile} options of \texttt{grgen.exe} (cf. \ref{grgenoptions}).

When profiling is turned on, the number of search steps carried out is printed to the console after each sequence execution.
A search step is a binding of a graph element to a pattern entity in case there are at least potentially several choices available.
Fetching an element by type chooses from all elements of that type in the graph, following an incident edge chooses from all incident edges of the corresponding node, and matching by storage access chooses from all elements in that storage, whereas getting the source or target node from an edge or mapping with a storage map just grabs the target element from the source element.
(Each binding from a choice counts as one step, the non-choice-bindings are not counted.)
Profiling the search steps allows you to assess the work needed for a transformation, to find the hot spots worth optimizing.
The less search steps are needed to find a match the better (we want to efficiently find patterns, want to minimize the amount of search needed to do so). 

The flags are applied when the actions are generated anew because the model or rule files changed. They are not when you just switch them in the shell script. It's your responsibility to delete the old generated dlls when you change those options (by switching them, or by introducing them differently from the compiler options used to generate the dlls)!

\begin{rail}
  'new' 'set' 'statistics' Filename
\end{rail}\ixkeyw{new}\ixkeyw{set}\ixkeyw{statistics}
Configures the compilation of the generated assemblies to use the statistics file specified, yielding pattern matchers adapted to the class of graphs described in that file.
Maps to the \texttt{-statistics} option of \texttt{grgen.exe} (cf. \ref{grgenoptions}, and see \ref{custom} on how to get such statistics).


%%%%%%%%%%%%%%%%%%%%%%%%%%%%%%%%%%%%%%%%%%%%%%%%%%%%%%%%%%%%%%%%%%%%%%%%%%%%%%%%%%%%%%%%%%%%%%%%
\section{Model and Graph Queries}

\begin{rail}
  'show' (() | 'num') ('nodes' (() | (() | 'only') NodeType) | 'edges' (() | (() | 'only') EdgeType))
\end{rail}\ixkeyw{show}\ixkeyw{num}\ixkeyw{nodes}\ixkeyw{edges}\ixkeyw{only}
Gets the \indexed{persistent name}s and the types of all the nodes/edges of the current graph.
If a node type or edge type is supplied, only elements compatible to this type are considered.
The \texttt{only} keyword excludes subtypes. Nodes/edges without persistent names are shown with a pseudo-name.
If the command is specified with \texttt{num}, only the number of nodes/edges will be displayed.

\begin{rail}
  'show' ('node' | 'edge') 'types'
\end{rail}\ixkeyw{show}\ixkeyw{node}\ixkeyw{edge}\ixkeyw{types}
Gets the node/edge types of the current graph model.

\begin{rail}
'show' ('node' ('super' | 'sub') 'types' NodeType | 'edge' ('super' | 'sub') 'types' EdgeType)
\end{rail}\ixkeyw{show}\ixkeyw{node}\ixkeyw{edge}\ixkeyw{super}\ixkeyw{sub}\ixkeyw{types}\indexmain{inheritance}
Gets the inherited/descendant types of \emph{NodeType}/\emph{EdgeType}.

\begin{rail}
  'show' ('node' 'attributes' (() | (() | 'only') NodeType) | 'edge' 'attributes' (() | (() | 'only') EdgeType))
\end{rail}\ixkeyw{show}\ixkeyw{node}\ixkeyw{edge}\ixkeyw{only}\ixkeyw{attributes}
Gets the available node/edge \indexed{attribute} types.
If \emph{NodeType}/\emph{EdgeType} is supplied, only attributes defined in \emph{NodeType}/\emph{EdgeType} are diplayed.
The \texttt{only} keyword excludes inherited attributes.\\
\begin{warning}
The \texttt{show nodes/edges attributes\dots} command covers types and \emph{inherited} types.
This is in contrast to the other \texttt{show\dots} commands where types and \emph{sub}types are specified or the direction in the type hierarchy is specified explicitly, respectively.
\end{warning}

\begin{rail}
 'show' ('node' Node | 'edge' Edge)
\end{rail}\ixkeyw{show}\ixkeyw{node}\ixkeyw{edge}
Gets the attribute types and values of a specific graph element.

\begin{rail}
  'show' GraphElement '.' AttributeName
\end{rail}\ixkeyw{show}
Displays the value of the specified attribute.

\begin{rail}
  'node' 'type' Node 'is' Node | 'edge' 'type' Edge 'is' Edge
\end{rail}\ixkeyw{node}\ixkeyw{edge}\ixkeyw{type}\ixkeyw{is}
Gets the information whether the first element is \indexed{type-compatible}\indexmainsee{compatible types}{type-compatible} to the second element.


%%%%%%%%%%%%%%%%%%%%%%%%%%%%%%%%%%%%%%%%%%%%%%%%%%%%%%%%%%%%%%%%%%%%%%%%%%%%%%%%%%%%%%%%%%%%%%%%
\section{Validation Commands}\label{sec:validate}

\GrG\ offers two different graph validation mechanisms,
the first checks against the connection assertions specified in the model,
the second checks against an arbitrary graph rewrite sequence containing arbitrary tests and rules.

\begin{rail}
  'validate' ('exitonfailure')? ('strict' ('only' 'specified')?)?
\end{rail}\ixkeyw{validate}\ixkeyw{exitonfailure}\ixkeyw{strict}\ixkeyw{only}\ixkeyw{specified}
Validates\indexmain{validate} if the current working graph fulfills the \indexed{connection assertion}s specified in the corresponding graph model (cf. ~\ref{sct:ConnectionAssertions}).
Validate without the strict modifier checks the multiplicities of the connections it finds in the host graph,
it ignores node-edge-node connections which are available in the host graph but have not been specified in the model.
The \emph{strict} mode additionally requires that all the edges available in the host graph must have been specified in the model.
This requirement is too harsh for models where only certain parts are considered critical enough to be checked
or might be a too big step in tightening the level of structural checking in an already existing large model.
So some form of selective strict checking is supported:
The \emph{strict only specified} mode requires strict matching (i.e. that all edges are covered) only of the edges for which connection assertions have been specified in the model.

\begin{rail}
  'validate' ('exitonfailure')? ('exec'|'xgrs') GRS
\end{rail}\ixkeyw{validate}\ixkeyw{exitonfailure}\ixkeyw{xgrs}\ixkeyw{exec}
Validates\indexmain{validate} if the current working graph satisfies the \indexed{graph rewrite sequence} given.
Before the graph rewrite sequence is executed, the instance graph gets cloned;
the sequence operates on the clone, allowing you to change the graph as you want to, without influence on the host graph.
Validation fails iff the sequence fails.
This gives a rather costly but extremely flexible and powerful mechanism to specify graph constraints.
The GrShell is exited with an error code if \texttt{exitonfailure} is specified and the validation fails.

\begin{example}
We reuse a simplified version of the road map model from Chapter~\ref{chapmodellang}:
\begin{grgen}
model Map;

node class city;
node class metropolis;

edge class street;
edge class highway
      connect metropolis [+] --> metropolis [+];
\end{grgen}
The node constraint on \emph{highway} requires all the metropolises to be connected by highways. Now have a look at the following graph:
\begin{center}
  \fbox{\includegraphics[width=8.5cm]{fig/map}}
\end{center}

This graph is valid but not strict valid.
\begin{grshell}
> validate
The graph is valid.
> validate strict only specified
The graph is NOT valid:
  CAE: city "Eppstein" -- highway "A3" --> metropolis "Frankfurt" not specified
> validate strict
The graph is NOT valid:
  CAE: city "Eppstein" -- highway "A3" --> metropolis "Frankfurt" not specified
  CAE: metropolis "Karlsruhe" -- street "trail" --> metropolis "Frankfurt" not specified
>
\end{grshell}
\end{example}


\pagebreak

%%%%%%%%%%%%%%%%%%%%%%%%%%%%%%%%%%%%%%%%%%%%%%%%%%%%%%%%%%%%%%%%%%%%%%%%%%%%%%%%%%%%%%%%%%%%%%%%
\section{Sequence Execution and Profiles}\label{grsthings}\indexmainsee{action}{graph rewrite sequence}

An \emph{action} denotes a graph rewrite rule.

\begin{rail}
  'show' 'profile' (Actionname)?
\end{rail}\ixkeyw{show}\ixkeyw{profile}
Shows the profile for the action specified by its name, or for all rules and tests in the rule set.

\begin{rail}
  GraphRewriteSequence: RewriteSequenceDefinition;
\end{rail}\ixkeyw{def}\indexmain{graph rewrite sequence definition}\indexmain{sequence definition}
This command allows to define a named sequence at runtime, for $RewriteSequenceDefinition$ have a look here  \ref{sec:sequencedefinition} in the rule application control language chapter.
Especially it allows to replace an old sequence definition, but only if the signature is identical.
Compiled sequences defined in rule files can't be replaced.
The defined sequence can then be used from following graph rewrite sequences (or following sequence definitions) in the shell.

\begin{example}
\begin{grgen}
# a sequence definition (of an interpreted sequence) is only available
# after it was registered the first time
# but it can get overwritten with a sequence of the same signature
# -> (self or mutually) recursive sequences must be constructed with empty body first
def chain(first:A):(last:A){ true }
def chain(first:A):(last:A){ if{(next:A)=chainPiece(first); (last)=chain(next); last=first} }
\end{grgen}
\end{example}

\makeatletter
\begin{rail}
  GraphRewriteSequence: ('exec'|'xgrs') SimpleRewriteSequence ;
\end{rail}\ixkeyw{exec}\ixkeyw{xgrs}\indexmain{graph rewrite sequence}\indexmainsee{GRS}{graph rewrite sequence}\ixnterm{GraphRewriteSequence}
This executes the graph rewrite sequence \emph{SimpleRewriteSequence}.
See Chapter~\ref{cha:xgrs} for graph rewrite sequences.
Additionally to the variable assignment in rule-embedded graph rewrite sequences, you are also able to assign \emph{persistent names} to parameters via  \texttt{Variable = \@(Text)}.

Graph elements returned by rules can be assigned to variables\indexmain{variable} using \texttt{(Para\-meters) = \emph{Action}}\indexmain{parameter}.
The desired variable identifiers have to be listed in \emph{Parameters}.
Graph elements required by rules must be provided using \texttt{Action (Para\-meters)}, where \emph{Parameters} is a list of variable identifiers.
For \indexed{undefined variables} see Section~\ref{ruledecls}, \emph{Parameters}.


%%%%%%%%%%%%%%%%%%%%%%%%%%%%%%%%%%%%%%%%%%%%%%%%%%%%%%%%%%%%%%%%%%%%%%%%%%%%%%%%%%%%%%%%%%%%%%%%
\section{Backend, Graph, and Actions Selection}\label{backend}\indexmain{action command}

\subsubsection*{Backend}
\GrG\ is built to support multiple backends implementing the model and action interfaces of libGr.
This is roughly comparable to the different storage engines MySQL offers.
Currently only one backend is available, the libGr search plan backend, or short LGSPBackend.

\begin{rail}
  'show' 'backend'
\end{rail}\nopagebreak\ixkeyw{show}\ixkeyw{backend}
List all the parameters supported by the currently selected backend.
The parameters can be provided to the \texttt{select backend} command.

\begin{rail}
  'select' 'backend' Filename ( ( ) | ':' Parameters )
\end{rail}\ixkeyw{select}\ixkeyw{backend}
Selects a \indexed{backend} that handles graph and rule representation.
\emph{Filename} has to be a .NET assembly (e.g.\ \texttt{lgspBackend.dll}\indexmain{LGSPBackend}).
Comma-separated \indexed{parameter}s can be supplied optionally; if so, the backend must support these parameters.
By default the LGSPBackend is used.


\subsubsection*{Graph}

\begin{rail}
  'select' 'graph' Graph
\end{rail}\ixkeyw{select}\ixkeyw{graph}
Selects the current \indexed{working graph}.
This graph acts as \emph{\indexed{host graph}} for graph rewrite sequences (see also Sections~\ref{ov:whatsallabout} and~\ref{grsthings}).
Though you can define multiple graphs, only one graph can be the active ``working graph''.

\begin{rail}
  'show' 'graphs'
\end{rail}\ixkeyw{show}\ixkeyw{graph}
Displays a list of currently available graphs.

\begin{rail}
  'delete' 'graph' Graph
\end{rail}\ixkeyw{delete}\ixkeyw{graph}
Deletes the graph \emph{Graph} from the backend storage.

\begin{rail}
  'custom' 'graph' ( ( ) | SpacedParameters )
\end{rail}\ixkeyw{custom}\ixkeyw{graph}
Executes a command specific to the current backend.
If \emph{SpacedParameters} is omitted, a list of available commands will be displayed (for the LGSP backend see Sections~\ref{custom}).


\subsubsection*{Actions}

\begin{rail}
  'select' 'actions' Filename
\end{rail}\ixkeyw{select}\ixkeyw{actions}
Selects a \indexed{rule set}.
\emph{Filename} can either be a .NET assembly (e.g.\ ``rules.dll'') or a source file (``rules.cs'').
Only one rule set can be loaded simultaneously.

\begin{rail}
  'show' 'actions'
\end{rail}\ixkeyw{show}\ixkeyw{actions}
Lists all the rules of the loaded rule set, their parameters, and their return values.
Rules can return a set of graph elements.

\begin{rail}
  'custom' 'actions' (() | SpacedParameters)
\end{rail}\ixkeyw{custom}\ixkeyw{actions}
Executes an action specific to the current \indexed{backend}.
If \emph{SpacedParameters} is omitted, a list of available commands will be displayed (for the LGSPBackend see Section~\ref{custom}).


%%%%%%%%%%%%%%%%%%%%%%%%%%%%%%%%%%%%%%%%%%%%%%%%%%%%%%%%%%%%%%%%%%%%%%%%%%%%%%%%%%%%%%%%%%%%%%%%
\section{LGSPBackend Custom Commands}
\label{custom}

%several rail nts are defines resolved to terminal with underscores

Don't be shy to use custom commands just because they are custom.
The search plan generation and explanation offered by them are of outstanding importance for achieving high performance solutions.
Leading directly to high-performance solutions by adapting the pattern matchers to the specifics of the host graph.
Or indirectly by explaining the chosen search plan so you can inspect it for the spots where the planner was not able to circumvent massively splitting passages; you may have to rethink your solutions regarding those, with maybe a change in the modelling, a change in the pattern, or even imperative code with hash set intersections/joins.

The \indexed{LGSPBackend} supports the following custom commands:

\begin{rail}
  'custom' 'graph' 'analyze'
\end{rail}\ixkeyw{custom}\ixkeyw{graph}\ixkeyw{analyze}
Analyzes\indexmain{analyzing graph} the current working graph.
The analysis data provides vital information for efficient \indexed{search plan}s.
Analysis data is available as long as \GrShell\ is running, i.e.\ when the host graph is manipulated, the analysis data is still available but outdated (which does not pose an issue unless the graph was changed massively regarding the relative number of elements per type or the connectedness-by-type relation).

\begin{rail}
  'custom' 'graph' 'statistics' 'save' Filename
\end{rail}\ixkeyw{custom}\ixkeyw{graph}\ixkeyw{statistics}\ixkeyw{save}
Write the statistics of the last analyze to the specified statistics file (the graph must have been analyzed before this command is called).
This way you can save the statistics of a characteristic graph of your domain and compile pattern matchers well adapted to those class of graphs straight from the beginning, saving you the costs of online analysis and matcher compilation.
See \ref{grgenoptions} or \ref{sec:compilerconfigshell} for explanations on how to do this.

\begin{rail}
  'custom' 'graph' setmaxmatches Number
\end{rail}\ixkeyw{custom}\ixkeyw{graph}\ixkeyw{setmaxmatches}
Sets the maximum amount of possible pattern matches to \emph{Number}.
This command affects the expression \texttt{[\emph{Rule}]}.
If \emph{Number} is less or equal to zero, the constraint is reset.

\begin{rail}
  'custom' 'graph' 'optimizereuse' BoolLit
\end{rail}\ixkeyw{custom}\ixkeyw{graph}\ixkeyw{optimizereuse}
If set to false it prevents deleted elements from getting reused in a rewrite (i.e. it disables a performance optimization).
If set to true, new elements may not be discriminable anymore from already deleted elements using object equality, hash maps, etc.

\begin{rail}
  'custom' 'actions' gensearchplan (Action*)
\end{rail}\ixkeyw{custom}\ixkeyw{actions}\ixkeyw{gensearchplan}
Creates a search plan (and executable code from it) for each rewrite rule \emph{Action} using the data from analyzing the graph (\texttt{custom graph analyze}).
Otherwise a \indexed{default search plan} is used.
If no rewrite rule is specified, all rewrite rules are compiled anew.
Analyzing and search plan/code generation themselves take some time, but they can lead to (massively) faster pattern matching, thus overall execution times;
the less uniform the type distribution and edge wiring between the nodes, the higher the improvements.
During the analysis phase the host graph must be in a shape ``similar'' to its shape when the main amount of work is done
(there may be some trial-and-error steps at different time points needed to get the overall most efficient search plan.)
A search plan is available as long as the current rule set remains loaded.
Specify multiple rewrite rules instead of using multiple commands for each rule to improve the search plan generation performance.

\begin{rail}
  'custom' 'actions' 'explain' Action
\end{rail}\ixkeyw{custom}\ixkeyw{actions}\ixkeyw{explain}
Shows the search plan currently in use for \emph{Action}, plus the subpatterns called by it.
This is an inspection tool comparable to the \texttt{explain} command offered by SQL-databases to inspect the search plans of their queries.
The explain command allows to evaluate the effects of performance optimization, esp. it allows to change the graph which serves as analysis data source for matcher regeneration, or to annotate the static rules with priorities (cf. \ref{annotations}), until a good search plan is built.
The search plan is shown as a list of search commands (with commands not doing real matching work shown in parenthesis), which is executed from top to bottom; for more on the search commands have a look at section \ref{searchplanning}.

\begin{rail}
  'custom' 'actions' dumpsourcecode BoolLit
\end{rail}\ixkeyw{custom}\ixkeyw{actions}\ixkeyw{dumpsourcecode}
If set to true, C\# files will be dumped for the newly generated searchplans (similar to the \texttt{-keep} option of the generator).


% todo: beispielshellscripte
