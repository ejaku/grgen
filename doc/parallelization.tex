\chapter{Parallelization}\label{cha:parallel} \indexmain{parallelization}

\GrG{} offers 3 ways of parallelization (hopefully performance-boosting):

\begin{itemize}
	\item you may endow rules with a parallelization annotation, then a parallelized version is generated, 
	\item you may employ parallelized graph isomorphy checking, 
	\item you may parallelize explicitly at sequence level (in contrast to the former internal implementation measures).
\end{itemize}


%%%%%%%%%%%%%%%%%%%%%%%%%%%%%%%%%%%%%%%%%%%%%%%%%%%%%%%%%%%%%%%%%%%%%%%%%%%%%%%%%%%%%%%%%%%%%%%%
\section{Internal parallelization}\label{sec:internalparallelization}

\subsubsection*{Action parallelization}

A rule or test annotated with \texttt{[parallelize=k]} will be matched with \texttt{k} worker threads from a thread pool.
More exactly: at most \texttt{k} worker threads, the number is clipped by the number of really available processors, on a single core the normal sequential matcher will be used.
(The current implementation defined maximum is 64.)

Parallelization distributes work alongside the first loop that is binding a pattern element to candidate graph elements.
If that loops only once nothing is gained.
If each loop iteration only executes very few search work following candidate assignment,
things become \emph{slower} because of threading and locking overhead.
Don't just append the parallelize annotation to each and every rule in the hope things will become faster!
Only search-intensive tasks benefit, only for them does the overhead of parallelization pay off.
But for those search-intensive tasks, you can achieve high speedups, reaping benefits from our-days multicore machines. 

Remark: Only the pattern matcher of an action is parallelized, so only the search process is parallelized.
This offers the biggest bang for the bucks invested, as search is \emph{the} expensive task, and it allows you to stick to the much simpler sequential programming model.
A parallelization of the kind presented in \cite{ParGraErs} offers potentially even higher speedups, but at the price of dealing with a parallel programming model, and at the price of graph partitioning that is very hard to get right in a general-purpose tool.

\begin{rail}
  FunctionParallelization: 'for' 'function' '[' 'parallelize' '=' BoolLit ']';
\end{rail}\ixnterm{FunctionParallelization}

In order to apply the \texttt{parallelize} annotation to an action, you need to first declare parallelization in the model with a \verb#for function[parallelize=true];# clause.
It causes the creation of (external) (method) functions that can be called from parallelized matchers (and implies "node edge unique;").
Actions with \texttt{parallelize} annotations causing the creation of parallelized matchers expect a "parallelized" model (functions callable from parallelized matchers).
An error is reported in case a parallelized rule/test is used with a model without a parallelize (function) declaration.

\begin{example}
You need a
\begin{grgen}
for function[parallelize=true];
\end{grgen}
in the model in order to use parallelized matchers in the rules:
\begin{grgen}
test testPar[parallelize=16]
{
	n1:Node --> n2:Node --> n3:Node --> n1;
}
\end{grgen}
\end{example}

\begin{rail}
  EqualsAnyParallelization: 'for' 'equalsAny' '[' 'parallelize' '=' Number ']';
\end{rail}\ixnterm{EqualsAnyParallelization}


\subsubsection*{equalsAny parallelization}

There's a second part that can be parallelized, the \texttt{equalsAny} and the \texttt{getEquivalent} functions checking for graph isomorphy of a new candidate graph against a set of known graphs.
The \emph{EqualsAnyParallelization} clause in the model must specify the number of worker threads used in parallel.
A \verb#for equalsAny[parallelize=8];# requests 8 worker threads for checking whether there's already a graph existing that is isomorphic to the candidate.
Graph isomorphy checking is expensive, you may gain considerable speed-ups in state space enumeration (at the price of having to maintain a set of already known graphs).

%%%%%%%%%%%%%%%%%%%%%%%%%%%%%%%%%%%%%%%%%%%%%%%%%%%%%%%%%%%%%%%%%%%%%%%%%%%%%%%%%%%%%%%%%%%%%%%%
\section{Sequence parallelization}\label{sec:sequenceparallelization}

Sequence parallelization is only available in the interpreted sequences, not in the compiled sequences (because it must be executed at top-level, outside of a rule, so outside of an embedded exec).

\subsubsection*{Parallel sequence}

\begin{rail}
  ParallelSequence: 'parallel' (ReturnAssignment)? (SequenceToBeExecutedInParallel + (','));
	SequenceToBeExecutedInParallel: 'in' SequenceExpression (',' SequenceExpression)? lbrace Sequence rbrace
\end{rail}\ixnterm{ParallelSequence}

The parallel sequence construct executes the sequences in parallel (given sufficient execution resources),
on the specified subgraphs, and evaluates the optional value expressions, making their result available as a local variable named \texttt{value} (in the sequences that are executed in parallel) optionally assigning the results of sequence execution.

\begin{example}
The example stemming from \texttt{tests/parallel/parallelWithValue.grs} shows how to generate Sierpinsky triangles in parallel in different (sub-)graphs, with an increasing number of generations (yielding different return values from the single sequences to be executed in parallel, assigning them after execution of the overall sequence).\\
\verb#exec parallel (::res1, ::res2, ::res3) = #\\
\verb#in ::g1.subgraph, 1 { init && (gen0* & gen1*)[value] ;> true },#\\
\verb#in ::g2.subgraph, 2 { init && (gen0* & gen1*)[value] ;> false },#\\
\verb#in ::g3.subgraph, 3 { init && (gen0* & gen1*)[value] ;> true }#
\end{example}

% TODO: example instead of this: Introduced a parallel (var1, var2, ...) = in subgraph-expr1, value-expr1 { seq1 }, in subgraph-expr2, value-expr2 { seq2 }, ... 

\begin{rail}
  ParallelArraySequence: 'parallel' 'array' (ReturnAssignment)? SequenceToBeExecutedInParallel;
\end{rail}\ixnterm{ParallelArraySequence}

Its sibling, the parallel array sequence construct executes the sequence in parallel (\texttt{subgraph\-array.size()} times, given sufficient execution resources), on the specified subgraphs, and optionally values, optionally assigning the results of parallel sequence execution.
The amount of subgraphs in the subgraphs array and values in the values array must be the same, the \verb#array<boolean># in the return-var will be of the same size.

\begin{example}
The example stemming from \texttt{tests/statespace\-Mapping/statespace\-Mapping\-Statespace\-Nodes\-Tree\-Parallel.grs} shows how to unfold a (tree-structured) state space (breadth-first) in parallel (in batches).\\
\verb#exec ::stopLevel=1 ;> for{step:int in [0 : ::stopLevel];#\\
\verb#parallel array in ::tobeprocessed { stateStep() } ;>#\\
\verb#{ ::tobeprocessed=::newlycreated.asArray(); ::newlycreated.clear() } }#
\end{example}\label{driverstatespace}

%TODO: example instead of this: Introduced a parallel array (var) = in subgraph-array-expr, value-array-expr { seq } - construct 

Both constructs are implemented with a thread pool, also using the parent(/main) thread for execution
in case the amount of threads from the thread pool is exceeded;
the single tasks in the parallel execution construct are executed in sequence
(by the parent thread) (at least one - the syntactically last - task is executed by the parent thread).

\begin{rail}
  SequenceParallelization: 'for' 'sequence' '[' 'parallelize' '=' Number ']';
\end{rail}\ixnterm{SequenceParallelization}

The \emph{SequenceParallelization} clause in the model allows to specify the number of worker threads in the sequences thread pool.
A \verb#for sequence[parallelize=8];# requests 8 worker threads for parallel sequence execution.

% Added debugger support with a choice of the sequence-to-be-debugged at construct execution begin

%%%%%%%%%%%%%%%%%%%%%%%%%%%%%%%%%%%%%%%%%%%%%%%%%%%%%%%%%%%%%%%%%%%%%%%%%%%%%%%%%%%%%%%%%%%%%%%%
\section{Locking}

While you can execute in parallel on independent graphs/resources, you have to synchronize access to shared graphs/resources.
Towards this purpose, some synchronization statements and methods are available in the sequences and rule language statements; they allow to implement critical sections.

\begin{rail}
  SequenceLock: 'lock' '(' SequenceExpression ')' lbrace Sequence rbrace;
\end{rail}\ixnterm{SequenceLock}

The specified synchronization object is entered/aquired when the sequence is entered and exited/released when the sequence is left (yielding the result of the contained sequences).

\begin{rail}
  LockStatement: 'lock' '(' Expression ')' lbrace Statements rbrace;
\end{rail}\ixnterm{LockStatement}

A construct with the same semantics is available in the rule language statements (acting as a pure statement without an implicit result value).

The procedures from the built-in package \texttt{Synchronization} allow to enter and exit lock objects.

\begin{description}
\item[\texttt{Synchronization::enter(.)}] enters the specified synchronization object (waiting for entry if it is already occupied by another thread (re-entering it in case it was already entered by the same thread)).
\item[\texttt{Synchronization::tryenter(.) : (boolean)}] trys to enter the specified synchronization object, returning false if it was already occupied, or enters it, returning true.
\item[\texttt{Synchronization::exit(.)}] exits the specified synchronization object (crashing if it was not already taken by the same thread, otherwise decreasing the entry counter, releasing other threads if the counter reaches 0).
\end{description}
   
\subsubsection*{getEquivalentOrAdd Locking}

The following two procedures can be employed in graph isomorphy comparison from parallel worker threads in a thread-safe (yet efficient) manner.

\begin{description}
\item[\texttt{getEquivalentOrAdd(.,.)}] returns the graph from the array of graphs 
  that is isomorphic to the graph given as first argument, or null if no isomorphic graph exists,
  in that case the graph is added to the array of graphs (to the rule language as well as the sequence computations).
\item[\texttt{getEquivalentStructurallyOrAdd(.,.)}] returns the graph from the array of graphs
  that is isomorphic neglecting attributes (so only structurally) to the graph given as first argument,
  or null if no isomorphic graph exists (even when neglecting attributes), in that case the graph is added to the array of graphs.
\end{description}
 
Both procedure(s) can be used concurrently from sequences executed in parallel,
the array is processed atomically and consistently, i.e. only one thread will add a graph
that is (not isomorphic to the existing ones yet) isomorphic to another graph concurrently processed by another thread (the other thread will then return this one);
it is (/they are) implemented in a way so that the majority of graph comparison work (and array reading) can occur in parallel, outside of array locking.

\begin{example}
The following example stems from \texttt{tests/statespace\-Mapping/statespace\-Mapping\-Statespace\-Nodes\-Graph\-Iso\-Check\-Parallel.grs}, which unfolds a (graph-structured) state space (breadth-first) in parallel (in batches).
It uses the same driver as example~\ref{driverstatespace}, but carries out a graph isomorphy comparison in the parallel worker threads, utilizing the \texttt{getEquivalentOrAdd} procedure in a \texttt{processState} subsequence, called from the \texttt{stateStep} main subsequence (backslashes separating lines removed for better readability).\\
\verb#def processState(state:graph, parent:StatespaceNode, red:set<graph>,#\\
\verb#                 green:set<graph>, blue:set<graph>)#\\
\verb#{#\\
\verb# equivalent:graph ;> { (equivalent) = getEquivalentOrAdd(state, ::statespace) } ;>#\\
\verb# if{ {{equivalent==null}} ;#\\
\verb#  child:StatespaceNode ;> lock(::newlycreated) { { ::newlycreated.add(state) } } ;>#\\
\verb#  lock(::root) { #\\
\verb#   (child)=::root.createAndLinkStatespaceNodeColored(parent, state, red, green, blue)#\\
\verb#  } ;>#\\
\verb#  lock(::graphToStatespaceNode) { { ::graphToStatespaceNode.add(state, child) } }#\\
\verb# ;#\\
\verb#  equivalentNode:StatespaceNode ;>#\\
\verb#  ( if{ lock(::graphToStatespaceNode) { { { equivalent in ::graphToStatespaceNode } } };#\\
\verb#        lock(::graphToStatespaceNode) {#\\
\verb#            { equivalentNode=::graphToStatespaceNode[equivalent] }#\\
\verb#        } ;> false; true }#\\
\verb#  )* ;>#\\
\verb#  lock(::root) { ::root.linkStatespaceNodeColored(parent, equivalentNode,#\\
\verb#                 state in red, state in green, state in blue) }#\\
\verb# }#\\
\verb#}#
\end{example}

%\pagebreak

