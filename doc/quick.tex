\chapter{Quickstart}\indexmain{quickstart}

\TODO{The most impressive features in five pages.}

In this chapter we'll build a \GrG\ system from scratch. 
We use it to construct non-deterministic state machines.
We further show the actual graph rewriting by removing $\varepsilon$-transitions from our state machines.
This is not too much about details but rather about the \GrG\ look and feel.

\section{Downloading \& Installing}
If you are reading this document, you probably did already download the \GrG\ software from our website (\url{http://www.grgen.net}).
Download the graph viewer \yComp\ as well.
Make sure you have the following system requirements installed
\begin{itemize}
	\item Java 1.5 or above
	\item Mono 1.2.3 on Unix-like platforms / .NET 2.0 or above on Microsoft Windows 
\end{itemize}
Unpack the packages to directories of your choice, for example into \texttt{/opt/grgen} and \texttt{/opt/ycomp}:
\begin{bash}
mkdir /opt/grgen /opt/ycomp
tar xvfj GrGenNET-V1_3_1-2007-12-06.tar.bz2
tar xvfj ycomp-1_3_4.tar.bz2
mv GrGenNET-V1_3_1-2007-12-06/* /opt/grgen/
mv ycomp-1_3_4/* /opt/ycomp/
rmdir GrGenNET-V1_3_1-2007-12-06 
rmdir ycomp-1_3_4
\end{bash}

evt. script in path

ycomp-connection

\section{Creating a Graph Model}
/home/grgen
graph model types for nodes, edges

\section{Creating some Graphs}
grshell

\section{The Rewrite Rules}

\section{Debugging and Output}
