\chapter{Sequence Graph Queries and Updates}\label{cha:graphquery}%\indexmain{sequence computations}\label{seqcomp}

In this chapter we take a look at sequence computations, which are not concerned with controlling rules, but with querying the graph (sequence expressions) or directly manipulating the graph (sequence statements).
Graph querying may occur in two ways, for one with global functions that return elementary data, for the other with pattern based queries (with rules/tests), that return an array of pattern matches.

%%%%%%%%%%%%%%%%%%%%%%%%%%%%%%%%%%%%%%%%%%%%%%%%%%%%%%%%%%%%%%%%%%%%%%%%%%%%%%%%%%%%%%%%%%%%%%%%
\section{Pattern Matching Based Queries}

\begin{rail}
  RuleQuery: '[' '?' Rule ']' ;
\end{rail}\ixnterm{RuleQuery}

The \emph{RuleQuery} clause applies a single rule or test.
Note that this is a \emph{Rule} as specified in Section~\ref{sec:ruleapplication}, that may receive input parameters, and may have filters applied to.
The filter calls (which allow e.g. to order the matches list and rewrite only the top ones, or to filter symmetric matches) are explained in \ref{sub:filters}.

The square braces (\texttt{[]}) and the \texttt{?} are part of the syntax, but act like the corresponding rule modifiers specified in \label{sec:ruleapplication}, switching for one a rule to a test, i.e.\ the rule application does not perform the rewrite part of the rule, and for the other searching for all pattern matches.

The execution semantics is: all matches of the rule pattern in the host graph are searched for, and returned as an \texttt{array} of \texttt{match<r>}.
For dedicated query purpose, a test is to be preferred, as the rewrite part won't be applied anyway. 
No values are returned in either case, and no return assignments specified.
The data of interest is to be extracted from the array of matches, or the match (type) respectively.
Towards this goal, esp. the array accumulation methods are to be applied, cf. Section~\ref{sec:arrayexpr}, and the filters before, cf. \ref{sub:filters}, as only the extract method is available from the array query methods.
No debugging modifier \texttt{\%} may be applied, rule queries are evaluated silently.

\begin{example}

  \begin{grgen}
node class namedObject
{
	name:string;
}

edge class WORKS_FOR;

edge class INTERESTED_IN
{
	weight : double;
}
  \end{grgen}

  \begin{grgen}
test interests(subject:namedObject<null>)
{
	subject -:INTERESTED_IN-> interest:namedObject;
---
	def var interestString:string = interest.name;
} \ keepOneForEach<interestString>

test weightedInterests(subject:namedObject<null>)
{
	subject -i:INTERESTED_IN-> interest:namedObject;
---
	def var weight:double = i.weight;
} \ orderDescendingBy<weight>

test sameCompanySharedInterest(subject:namedObject)
{
	subject -:WORKS_FOR-> company:namedObject <-:WORKS_FOR- person:namedObject;
	iterated it {
		subject -:INTERESTED_IN-> interest:namedObject <-:INTERESTED_IN- person;
	}
---
	def var score:double = yield([?it].extract<interest>().size());
	def ref interests:array<string> = yield([?it].extract<interest>().extract<name>());
} \ orderDescendingBy<score>

test sameCompanySharedWeightedInterestClipped(subject:namedObject, var clipAmount:int)
{
	subject -:WORKS_FOR-> company:namedObject <-:WORKS_FOR- person:namedObject;
	iterated it {
		subject -is:INTERESTED_IN-> interest:namedObject <-ip:INTERESTED_IN- person;
	---
		def var sharedWeightedInterest:double = Math::min(is.weight, ip.weight);
	} \ orderDescendingBy<sharedWeightedInterest>
---
	iterated it\orderDescendingBy<sharedWeightedInterest>\keepFirst(clipAmount);
	def var score:double = yield([?it].extract<sharedWeightedInterest>().sum());
	def var interestsString:string = yield([?it].extract<interest>().extract<name>().asString(","));
} \ orderDescendingBy<score>

test fetchByName(var name:string) : (namedObject)
{
	namedObj:namedObject;
	if{ namedObj.name == name; }
	return(namedObj);
}
  \end{grgen}\label{exsimplequeryrules}
\end{example}


\begin{example}\label{exsimplequery}

Let us take a look at some example queries, based on Example~\ref{exsimplequeryrules}, which in turn is based on a test that you can find under \texttt{tests/queries} -- also including some example data.

In the model we introduce a "generic" namedObject used for subjects, i.e. persons, and objects like companies or hobbies, and edges showing employment and interest for subjects (the model shows only a few general types, which is untypical for GrGen, but common for modelling with Neo4j/Cypher where the idea for this example stems from).

We start with some simple queries, asking for the interests of a subject, and the intensity of the interest.
In the rules file, they show up as tests containing the patterns to query for.
Then we continue with some more complex queries, asking for the shared interests of persons working in the same company, and the intensity of the shared interest.
In the rules file, they again show up as tests containing the patterns to query for.

\vspace{3mm}

In order to obtain all interests (\texttt{namedObject}s) all persons are interested in, we drop the following query in \GrShell:

	\begin{grshell}
eval [?interests(null)\keepOneForEach<interestString>]
  \end{grshell}

It fetches for an unspecified subject (\texttt{null}, thus for all subjects, due to the \texttt{null} specifier at the parameter in the test, which causes searching) all interests, in the form of \texttt{array<match<interests>>}. The array is freed from duplicates within the \texttt{interestString} member, so only matches with distinct \texttt{interestString}s remain (the first distinct value from the member in the array is kept).

In order to obtain the overall intensity of interests of a single person, we use the following query in \GrShell:

	\begin{grshell}
eval [?weightedInterests(::subject)].extract<weight>().sum()
  \end{grshell}

From the array of matches is an array of the match members \texttt{weight} of type \texttt{array<double>} extracted (resulting from the equivalently named pattern element), and then reduced to a single scalar value with the array accumulation method \texttt{sum}.

In order to obtain the interests of two persons, we use the following query in \GrShell:

	\begin{grshell}
eval [?interests(::subject)].extract<interestString>().asSet() | [?interests(::anotherSubject)].extract<interestString>().asSet()
  \end{grshell}
	
It fetches the interests of \texttt{::subject} and \texttt{::anotherSubject}, in the form of \texttt{array<match<interests>>}, extracts the \texttt{interestString}s, resulting in two \texttt{array<string>}, transforms them to sets (of type \texttt{set<string>}), and finally joins them with set union \texttt{|}.

\end{example}


\begin{example}
Let us continue Example~\ref{exsimplequery}. In order to obtain the persons working in the same company with the highest amount of shared interests, we drop the following query in \GrShell:

  \begin{grshell}
eval [?sameCompanySharedInterest(::subject)\orderDescendingBy<score>\keepFirst(3)]
  \end{grshell}

It searches for all matches of pattern \texttt{sameCompanySharedInterest} for the given subject, in the form of \texttt{array<match<sameCompanySharedInterest>>}.
The array is then ordered descendingly alongside \texttt{score} (which just gives the amount of shared interests), and clipped to the first 3 matches, thus persons.
Besides the order-by known from other query languages that shows up directly in the query expression, is a kind of group-by employed, in the test used, with a common pattern giving the shared part, and differences encapsulated in the iterated (we don't obtain a cartesian product of all matches and then group afterwards to factor out common parts, but match common parts and extend them by the differences to give the complete match).
The interesting information from the iterated is extracted and accumulated with an iterated query (cf. \ref{sec:primexpr}), which returns an array of the matched iterated pattern (\texttt{array<match<sameCompanySharedInterest.it>>}).
(It can be seen as an get-all-matches rule query inside a rule, and an iterated as a nested all-bracketed rule).

We improve upon the last query by obtaining the persons working in the same company with the highest intensity of interest regarding their shared interests:

  \begin{grshell}
eval [?sameCompanySharedWeightedInterestClipped([?fetchByName("hans")][0].namedObj, ::clipAmount)]
  \end{grshell}
	
It employs a subquery \texttt{fetchByName} to obtain a dedicated person, i.e. \texttt{namedObject} by name, reducing the resulting array of matches first to a single array element (the first one) and then to a single member of the match (using the name from the named graph would be preferable in a model fully fitting to the \GrG-technological space); utilizing a query to compute an argument is also of interest in regular rule applications.
Furthermore, it computes the shared interest in an object as the smaller of the interests of both subjects in the object, accumulating over all objects the subjects are interested in, with a cut-off applied by a clipAmount.
\end{example}

\subsubsection*{Multi Rule Query}

\begin{rail}
  MultiRuleQuery: '[' '?' '[' (Rule + ',') ']' (MatchClassFilterCalls)? (MatchClassSpec)? ']' ;
	MatchClassSpec: backslash '<' 'class' MatchClassIdent '>' ;
\end{rail}\ixnterm{MultiRuleQuery}\ixnterm{MatchClassSpec}

The \emph{MultiRuleQuery} clause applies multiple (distinct) rules or tests, that all implement the same match class, cf. Section~\ref{sec:matchclass}).
It defines a query version of the \emph{MultiRuleExecution}, cf. Section~\ref{sec:multiruleapplication}.
It does not allow return assignments in contrast to that one, as with all pattern-based query constructs only the pattern elements -- but all of them -- are accessible.
In contrast to the \emph{MultiRuleExecution}, a \emph{MatchClassSpec} is required. 
It defines the type of the result (it cannot be derived from the rules, as these may implement multiple match classes).

\begin{example}
The rules from Example~\ref{exmatchclass} can be applied as a query with e.g. \\
\verb#[?[r,s]\Score.orderDescendingBy<score>\Score.keepFirst(3)\<class Score>]#.
\end{example}


%%%%%%%%%%%%%%%%%%%%%%%%%%%%%%%%%%%%%%%%%%%%%%%%%%%%%%%%%%%%%%%%%%%%%%%%%%%%%%%%%%%%%%%%%%%%%%%%
\section{Graph and Subgraph Based Queries and Updates}\label{sec:queryupdate}\label{sec:visited}

The graph and subgraph oriented parts of the sequence expressions are built from four groups,
the procedures for basic graph manipulation, the functions for querying the graph structure, the functions and procedures of the subgraph operations, and the visited flag query, assignment, and procedures.

The first group is built from basic graph manipulation operators, as defined in \ref{procstab} and described in \ref{sub:procedures}.
Elements may be added, removed, or retyped, and nodes may be merged or edges redirected.
Not available are numerical functions, they are only offered by the computations of the rules. 

\begin{example}
The sequence computation \verb#{ (::x)=add(N); (::x)=retype(::x,M); rem(::x) }# adds a newly created node of type \texttt{N} to the graph (storing it in the global variable \verb#::x#), retypes it to \texttt{M}, and finally removes it again from the graph.
\end{example}

The second group is built from the operators querying primarily the connectedness of graph elements,
as defined in \ref{sub:functions}.
You may ask for one for the nodes or edges of a type.
You may query for the other for the source or target or opposite node of an edge.
Furthermore, you may query for adjacent nodes and incident edges,
maybe even transitively for the reachability.
Furthermore you may ask with a predicate whether nodes or edges are adjacent or incident to other nodes or edges, maybe even transitively for reachability.

\begin{example}
\verb#for{x:N in nodes(N); for{::y in outgoing(x); { ::z=target(::y); ::z.a = 42 } } }# 
is a sequences that sets the attribute \texttt{a} to 42 for all nodes that are adjacent as targets to a source node \texttt{x} of type \texttt{N}.
You will receive a runtime exception if the type of \verb#::z# does not possess an attribute \texttt{a}.

A more realistic example is to check whether two nodes returned by some rule applications are reachable from each other, carrying out a change only in this case:\\
\verb#(::x)=r() ;> (::y)=s() ;> if{ {{isReachable(::x,::y)}} ; doSomething(::x, ::y) }#
\end{example}

The third group is defined by functions and procedures that operate on (sub-)graphs, as defined in \ref{sub:functions} and \ref{sub:procedures}.
They are especially useful in state space enumeration, cf. \ref{sec:statespaceenum}.
To this end, parallelized graph isomorphy checking with the \texttt{equalsAny} function is especially of interest.
You may import, clone, or compute induced subgraphs.
You may export a subgraph or insert a subgraph into the hostgraph.

\begin{example}
\verb#( doSomething() ;> { File::export("graph"+i+".grs") } )*# is a sequence scheme for exporting a graph after each iteration step in a loop, gaining a series of snapshots on the hard drive.
In a later step, you may then conditionally add exported graphs to the host graph: 
\verb#if{cond; { ::g=File::import("graph"+n.a+".grs"); insert(::g) } }#
\end{example}

\begin{example}
When you model a state space with \texttt{Graph} representative nodes (\emph{not} \texttt{graph} standing for a real (sub)graph), which are pointing with \texttt{contains} edges to the nodes contained in their state (i.e. subgraph), and store additionally a replica of the subgraph in a \texttt{sub} attribute of the \texttt{Graph} node, so it is readily available for comparisons,
then the step of a state space enumeration with isomorphic state pruning is controlled with code like this:\\
\verb#<< modifyCurrent(gr) ;; {adj=adjacent(gr, contains); sub=inducedSubgraph(adj)}#\\
Inside the backtracking double angles, a new state is computed as first step by modifying the currently focused state received as input \texttt{gr:Graph} from the previous step. The modified subgraph is extracted for comparison by computing the \texttt{inducedSubgraph} from the nodes \texttt{adjacent} via \texttt{contains}-edges to the \texttt{gr}-node.\\
\verb#;> for{others:Graph in nodes(Graph); {{sub!=others.sub}} } && #\\
The extracted subgraph is compared with all already enumerated subgraphs that can be accessed by their \texttt{Graph} representative node. Only if none is isomorphic to it, do we continue with making the state persistent.\\
\verb#/ {(ngr)=insertInduced(adj, gr)} && link(gr,ngr) && {ngr.sub=sub} /#\\
During a backtracking pause, the modified subgraph is cloned and inserted flatly into the host graph again with \texttt{insertInduced}. A link is added from the old representative to this new representative, to reflect ancestry. Then the subgraph attribute of the new representative \texttt{ngr} is filled with the previously computed subgraph \texttt{sub}. Remark: the first \texttt{inducedSubgraph} above does not contain the representative node and thus is missing all containment edges, too. This \texttt{insertInduced} includes the representative node and thus the containment edges. Syntactical remark: \texttt{inducedSubgraph} is used in an assignment with a function call as RHS, whereas \texttt{insertInduced} is employed from a procedure call which requires parenthesis around the output arguments.\\
\verb#&& stateStep(ngr, level+1) >>#\\
Finally, we continue state space construction with the next step, modifying the just inserted subgraph.
After this step returns (with \texttt{false} as result), do the backtracking double angles roll back the modification -- keeping the changes written during the pause untouched -- and execute \texttt{modifyCurrent} on the next match available in \texttt{gr}.
\end{example}

The fourth group are the visited flags related operations,
as described in chapter \ref{sub:visitedaccess}.
Available is an expression for reading a visited flag, an assignment for writing a visited flag, and procedures for managing the visited flags as defined in \ref{procstab}.

\begin{example}
Because of the need to allocate and deallocate them, the visited flags are typically used with code like this:
\verb#flag:int ;> {(flag)=valloc()} ;> r(flag) ;> {vfree(flag)}#\\
In addition, they may be read in the sequence expressions, and written in the sequence computations:
\verb#if{ {{!n.visited[flag]}} ; { n.visited[flag] = true } }#
\end{example}

In the sequences only the sequence expressions are available to compute the parameters for the functions and procedures, compared to the full-fledged expressions of the computations language.

%\pagebreak


%%%%%%%%%%%%%%%%%%%%%%%%%%%%%%%%%%%%%%%%%%%%%%%%%%%%%%%%%%%%%%%%%%%%%%%%%%%%%%%%%%%%%%%%%%%%%%%%
\section{Query Looping and Counting Rule Execution}

Instead of querying the graph with a sequence expression, and continuing processing with filters and array accumulation methods, is it possible to query it with a rule in a for loop of the sequences, processing the single matches one-by-one.

\begin{rail}
  ExtendedControl:
    'for' lbrace Variable ':' Type
    ('in' '[' '?' RuleIdent ']' ';')\\
    RewriteSequence rbrace
    ;
\end{rail}\ixkeyw{for}\label{formatch}

The \texttt{for} matches loop allows to iterate through the matches found for an all-bracketed rule reduced to a test (or a test); i.e. the rule is not applied, we only iterate its matches.
The loop variable must be of type \texttt{match<r>} with \texttt{r} being the name of the rule matched (the type must be statically known).
The elements (esp. the nodes and edges) of the pattern of the matched rule can then be accessed by applying the \texttt{.}-operator on the loop variable, giving the name of the element of interest after the dot.
Note: the elements must be assigned to a variable in order to access their attributes, a direct attribute access after the match access is not possible.
Note: the match object allows only to access the top level nodes, edges, or variables.
If you use subpatterns or nested patterns and want to access elements found by them, you have to \texttt{yield}(\ref{sub:yield}) them out to the top-level pattern.

In contrast to a rule query that only allows to fetch data from the graph without modifying it, is it possible for a counted rule application to apply a rule (including its rewrite part), and to assign the size of the matches array as rudimentary query result to a variable.

\begin{rail}
  RuleExecution: (() 
	| '(' (Variable+',') ')' '=') \\ ('count' '[' RuleModifier Rule ']' '=''>' Variable);
\end{rail}\ixnterm{RuleExecution}

The \texttt{count}ed all bracketing \texttt{count[r]=>c} assigns the \indexed{count} of matches of rule \texttt{r} to the variable \texttt{c}, and applies \texttt{r} on all the matches.
With \texttt{count[?r]=>c} the matches are only counted, no rewrites are carried out.

% todo: beispiele im text bringen
