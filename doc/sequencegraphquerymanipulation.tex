\chapter{Graph Queries}\label{cha:graphquery}
%\chapter{Sequence Graph Queries and Updates}\label{cha:graphquery}%\indexmain{sequence computations}\label{seqcomp}
% match type würde zwar konzeptionell hier passen, aber schon in filtering/ordering und for-matches-loop in sequence computations gebraucht.

In this chapter we take a look at pattern queries -- the most powerful and flexible way of querying the graph.
They are more powerful than the simple global functions returning elementary data (described in Section~\ref{sec:queryupdate}),
and more flexible than the returning of pre-selected values from rules or tests (described in Chapter~\ref{chaprulelang}).

Pattern queries employing tests or rules as tests in order to obtain some result value define the second way of pattern usage beside the already known one employing rules to carry out host graph state changes.
%Here we take a look at the secondary usage of patterns besides the already known transformation workflow where they are employed with rules to carry out host graph state changes.
%for the the query workflow, the directly following array accumulation methods are of importance. 

\subsubsection*{Query Workflow Compared to Transformation Workflow}

%The most direct and powerful way are the pattern based queries (employing rules/tests), which return an array of pattern matches for further processing with filters and array accumulation functions.

The heart and soul of \GrG{} is pattern matching -- it is utilized from two different workflows.
Both start the same way, the pattern is matched, yielding an \emph{array} of \emph{match} objects, which gets further processed by \emph{filters}.
\begin{description}
	\item[rule application] The resulting matches are used to apply the rewrite part of the rule on the resulting matches, causing a graph state change, ending in a boolean value telling about rule application success (this can be seen as a built-in automatically applied reduction) %the matches are only accessible for the rewriting parts
	\item[querying] The resulting matches are fed into post-matches-accumulation methods, that distill a scalar value out of them, or an array or set of interesting values; the matches may be also reported directly instead (no graph state change is carried out when working with queries).
\end{description}

\begin{tikzpicture}[baseline=(T.base)] \tt
	\begin{scope}[minimum size=0.5cm]
		\tikzstyle{every node}=[draw]
		\node (pm)     at (4   ,10) {\texttt{pattern matching}};
		\node (ma)     at (4   ,8) {\texttt{a:array<match<r>>}};
		\node (fo)     at (9   ,8) {\texttt{filtering/ordering}};
		\node (rw)     at (0   ,6) {\texttt{rewriting/graph state change}};
		\node (ar)     at (0   ,4) {\texttt{a.size() > 0}};
		\node (rb)     at (0   ,2) {\texttt{r:boolean}};
		\node (aa)     at (9   ,6) {\texttt{array(-accumulation)-methods}};
		\node (r)     at (9   ,4) {\texttt{r:array<T> or r:set<T> or r:T}};
	\end{scope}
	\draw[thick,-open triangle 45]  (pm) -> (ma)  ;
	\draw[thick,-open triangle 45]  (ma) -> (fo)  ;
	\draw[thick,-open triangle 45]  (fo) -> (ma)  ;
	\draw[thick,-open triangle 45]  (ma) -> (rw)  ;
	\draw[thick,-open triangle 45]  (rw) -> (ar)  ;
	\draw[thick,-open triangle 45]  (ar) -> (rb)  ;
	\draw[thick,-open triangle 45]  (ma) -> (aa)  ;
	\draw[thick,-open triangle 45]  (aa) -> (r)  ;
\end{tikzpicture}

In order to manipulate a graph you need rules, for querying you can use rules as well as tests, but the more natural partner for querying are tests (coming without the specification of a graph change).
The advanced array operations employed in the queries allow to flexibly pick at the call site (outside of the rule/test) the values of interest as well as the accumulation method, instead of having to encode them in the rule, and allow to accumulate the data from multiple matches (typically obtained from different spots) directly, instead of having to work with some temporary accumulation variables (but only in case a method suiting your needs is available).
The same holds for iterated queries inside a rule (allowing to accumulate the data from multiple matches of an iterated pattern, while some other yielding constructs allow to accumulate bottom-up in case no method is available or a nested iterated requires value accumulation, as of now only one level of iterated query filtering and post-matches-accumulation is supported; see Chapter~\ref{cha:yielding} for more).

In the following section we introduce the post-matches-accumulation methods/advanced array operations (these are predominantly employed in the query workflow).

\section{Advanced Array Operations (Querying Support)}\label{sec:accumulation}\indexmain{accumulation}

The following array methods can be employed in the rule language and in the sequence expressions.
%From the array methods specifying and accessing a member of the array value type, the \texttt{extract<member>()} method is supported as well as the ordering methods \texttt{order\-Ascending\-By<member>()} and \texttt{order\-Descending\-By<member>()}, plus \texttt{group\-By<member>()} and \texttt{keep\-One\-For\-Each<member>()} (note that further means of ordering are available as filter calls).
We distinguish the 
\begin{itemize}
	\item array accumulation methods
	\item by-member-or-attribute access methods
	\item per-array-element with lambda expression access methods
\end{itemize}

\subsection{Array Accumulation Methods} 

Some accumulation method calls are available on arrays of esp. numbers:

\begin{description}
\item[\texttt{.sum()}] returns the sum of the elements in the array, as \texttt{T'} for \texttt{array<T>}
\item[\texttt{.prod()}] returns the product of the elements in the array, as \texttt{T'} for \texttt{array<T>}
\item[\texttt{.min()}] returns the minimum of the elements in the array, as \texttt{T'} for \texttt{array<T>}
\item[\texttt{.max()}] returns the maximum of the elements in the array, as \texttt{T'} for \texttt{array<T>}
\item[\texttt{.avg()}] returns the average (mean) of the elements in the array, as \texttt{double}
\item[\texttt{.med()}] returns the median of the elements in the array that must be ordered, as \texttt{double}
\item[\texttt{.medUnordered()}] returns the median of the elements in the array, as \texttt{double}
\item[\texttt{.var()}] returns the variance of the elements in the array, as \texttt{double}
\item[\texttt{.dev()}] returns the deviation of the elements in the array, as \texttt{double}
\item[\texttt{.and()}] returns the conjuntion of the elements in the array (of \texttt{boolean} values), as \texttt{boolean}, \texttt{true} in case of an empty array
\item[\texttt{.or()}] returns the disjunction of the elements in the array (of \texttt{boolean} values), as \texttt{boolean}, \texttt{false} in case of an empty array
\end{description}\label{arrayaccumulationmethod}

The input type \texttt{T} must be a numeric type, i.e. one of the integer types \texttt{byte} or \texttt{short} or \texttt{int} or \texttt{long} or one of the floating point types \texttt{float} or \texttt{double}. 
The result type \texttt{T'} is \texttt{int} for input types \texttt{byte} or \texttt{short} or \texttt{int}, it is \texttt{long} for input type \texttt{long}, it is \texttt{float} for input type \texttt{float}, and it is \texttt{double} for input type \texttt{double} (the same type is used for carrying out the accumulation, defining bounds and precision).

Note that also the array methods introduced in Section~\ref{sec:arrayexpr} can be used for accumulation (/towards query purpose), e.g. the \texttt{.asString(separator)} method (or the \texttt{.asSet()} method).

\subsection{Array By-Member Methods} 

Several by-member access (or array query) methods are available on 

\begin{itemize}
	\item arrays of node or edge types bearing attributes
	\item arrays of match types having match member entities
	\item arrays of iterated match types having match member entities
	\item arrays of match class types having match class member entities
\end{itemize}

They inspect the value of the attribute \texttt{attr} or member of the graph element or internal class object or match stored in the array, instead of the value contained in the array directly.

\begin{description}
\item[\texttt{.indexOfBy<attr>(valueToSearchFor)}] returns the first position \texttt{valueToSearchFor:T} appears at, as \texttt{int}, or -1 if not found.
\item[\texttt{.indexOfBy<attr>(valueToSearchFor, startIndex)}] returns the first position \texttt{valueToSearchFor:T} appears at (moving to the end), when we start the search for it at array position \texttt{startIndex:int}, as \texttt{int}, or -1 if not found.
\item[\texttt{.lastIndexOfBy<attr>(valueToSearchFor)}] returns the last position \texttt{valueToSearchFor:T} appears at, as \texttt{int}, or -1 if not found.
\item[\texttt{.lastIndexOfBy<attr>(valueToSearchFor, startIndex)}] returns the last position \texttt{valueToSearchFor:T} appears at (moving to the begin), when we start the search for it at array position \texttt{startIndex:int}, as \texttt{int}, or -1 if not found.
\item[\texttt{.indexOfOrderedBy<attr>(valueToSearchFor)}] returns a position where \texttt{valueToSearchFor:T} appears at, as \texttt{int}, or -1 if not found. The array must be ordered, otherwise the results returned by the binary search employed will be wrong; in case of multiple occurences, an arbitrary one is returned.
\end{description}

\noindent The search methods just introduced return the index of a found element, the following ordering or grouping methods return an array.

\begin{description}
\item[\texttt{.extract<attr>()}] returns for an \texttt{array<T>} an \texttt{array<S>}, where \texttt{S} is the type of the attribute or member \texttt{attr} to be extracted (or projected out).
\item[\texttt{.orderAscendingBy<attr>()}] returns an array with the content of the input array ordered ascendingly; available for an attribute or member \texttt{attr} of an orderable basic type.
\item[\texttt{.orderDescendingBy<attr>()}] returns an array with the content of the input array ordered descendingly; available for an attribute or member \texttt{attr} of an orderable basic type.
\item[\texttt{.groupBy<attr>()}] returns an array with the content of the input array grouped alongside the attribute (equal values being neighbours); available for an attribute or member \texttt{attr} of basic type or a node or edge type.
\item[\texttt{.keepOneForEach<attr>()}] returns an array with the content of the input array freed from duplicates; available for an attribute or member \texttt{attr} of basic type or a node or edge type.
\end{description}\label{arraybymemberaccessmethod}
%And again: the \verb#X_by<attr>#-versions inspect the value of the attribute of the graph element stored in the array.

The ordering or access-alongside-order methods utilize the ordering relation, so they support only members or attributes of numeric type (\texttt{byte}, \texttt{short}, \texttt{int}, \texttt{long}, \texttt{float}, \texttt{double},) or \texttt{string} or \texttt{enum} type -- also \texttt{boolean} is supported, even if not of much use).
The other query methods utilize equality comparisons, so they support additionally members or attributes of node or edge type.

\subsection{Array Per-Element Methods with Lambda Expressions} 

Some per-element array methods are available with a special syntax giving a lambda expression (instead of arguments).
The lambda expression is evaluated for each element of the array, with the current array element being available in a to be declared element variable (of type \texttt{T} for an array of type \texttt{array<T>}), and optionally the array index in a to be declared index variable (of type \texttt{int}).
A type may be given enclosed in angles to denote a mapping result type.

\begin{rail}
  MethodSelector: '.' FunctionIdent ('<' Type '>' | ) lbrace PerElementClause rbrace;
  PerElementClause: (IdentDecl ':' Type '->')? IdentDecl ':' Type '->' Expression;
\end{rail}\ixnterm{MethodSelector}\ixnterm{PerElementClause}

\begin{description}
\item[\PVerb{.map<S>{element:T -> lambda-expr}}] returns a new \texttt{array<S>},
   created by mapping each \texttt{element:T} (the name can be chosen freely) of the original \texttt{array<T>} with the lambda expression (of type \texttt{S}) to its resulting counterpart.
\item[\PVerb{.removeIf{element:T -> lambda-expr}}] returns a new \texttt{array<T>},
   created by removing each \texttt{element:T} (the name can be chosen freely) of the original \texttt{array<T>} for which the predicate given by the lambda expression (which must be of type \texttt{boolean}) yielded true.
\item[\PVerb{.map<S>StartWith{init-expr}AccumulateBy}]
\PVerb{{accumulation-var:T, element:T -> lambda-expr}} returns a new \texttt{array<S>},
   created by mapping each \texttt{element:T} (the name can be chosen freely) of the original \texttt{array<T>} with the lambda expression (of type \texttt{S}) to its resulting counterpart.
	The lambda expression is also fed the previous mapping result, for the first iteration initialized by \texttt{init-expr}.
\end{description}

All methods allow for an optional index: the variable \texttt{index:int} (the name can be chosen freely) iterates the array indices [0,a.size()-1] of the corresponding elements in the array \texttt{a:array<T>} in the\\
\verb#a.map<S>{index:int -> element:T -> lambda-expr}# and\\
\verb#a.removeIf{index:int -> element:T -> lambda-expr}# and\\
\verb#a.map<S>StartWith{init-expr}AccumulateBy#\\
\verb#{accumulation-var:T, index:int -> element:T -> lambda-expr}#
expressions.

Even more, they all allow for an optional array access variable: \verb#this_:array<T># given as first variable declaration in the lambda expression (the name can be chosen freely), separated by a semicolon, references the original input array, allowing to access it in order to obtain further information, in the\\
\verb#a.map<S>{this_:array<T>; element:T -> lambda-expr}# and\\
\verb#a.removeIf{this_:array<T>; element:T -> lambda-expr}# and\\
\verb#a.map<S>StartWith{this_:array<T>; init-expr}AccumulateBy#\\
\verb#{this_:array<T>; accumulation-var:T, index:int -> element:T -> lambda-expr}# expressions.

This allows to e.g. decide based on a comparison against the average of an attribute: \verb#[a.removeIf{this_:array<T>; element:T -> element.a < this_.extract<a>().avg()}])#

%%%%%%%%%%%%%%%%%%%%%%%%%%%%%%%%%%%%%%%%%%%%%%%%%%%%%%%%%%%%%%%%%%%%%%%%%%%%%%%%%%%%%%%%%%%%%%%%
\section{Pattern Matching Based Queries}\label{sec:patternbasedgraphquery}

\begin{rail}
  RuleQuery: '[' '?' Rule ']' ;
\end{rail}\ixnterm{RuleQuery}

The \emph{RuleQuery} clause applies a single rule or test.
Note that this is a \emph{Rule} as specified in Section~\ref{sec:ruleapplication}, that may receive input parameters, and may have filters applied to.
The filter calls (which allow e.g. to order the matches list and rewrite only the top ones, or to filter symmetric matches) are explained in Chapter~\ref{sub:filters}.

The square braces (\texttt{[]}) and the \texttt{?} are part of the syntax, but act like the corresponding rule modifiers specified in Section~\ref{sec:ruleapplication}, switching for one a rule to a test, i.e.\ the rule application does not perform the rewrite part of the rule, and searching for the other for all pattern matches.

The execution semantics is: all matches of the rule pattern in the host graph are searched for, and returned as an \texttt{array} of \texttt{match<r>}.
For query purpose, a \texttt{test} is to be preferred over a \texttt{rule}, as the rewrite part won't be applied anyway. 
No values are returned in either case, and no return assignments specified.
The data of interest is to be extracted from the array of matches, or the match (type) respectively.
Towards this goal, the filters are to be applied, cf. Chapter~\ref{sub:filters}, but esp. the array accumulation and query methods introduced earlier in this chapter, as well as in Section~\ref{sec:arrayexpr}, notably the \texttt{extract} method.
No debugging modifier \texttt{\%} may be applied, rule queries are evaluated silently.

Note that the rule query is a sequence expression -- it cannot be applied in a sequence in the place of a rule.

\begin{example}
  \begin{grgen}
node class namedObject
{
	name:string;
}

edge class WORKS_FOR;

edge class INTERESTED_IN
{
	weight : double;
}
  \end{grgen}
\end{example}


\begin{example}

  \begin{grgen}
test interests(subject:namedObject<null>)
{
	subject -:INTERESTED_IN-> interest:namedObject;
---
	def var interestString:string = interest.name;
} \ keepOneForEach<interestString>

test weightedInterests(subject:namedObject<null>)
{
	subject -i:INTERESTED_IN-> interest:namedObject;
---
	def var weight:double = i.weight;
} \ orderDescendingBy<weight>

test sameCompanySharedInterest(subject:namedObject)
{
	subject -:WORKS_FOR-> company:namedObject <-:WORKS_FOR- person:namedObject;
	iterated it {
		subject -:INTERESTED_IN-> interest:namedObject <-:INTERESTED_IN- person;
	}
---
	def var score:double = yield([?it].extract<interest>().size());
	def ref interests:array<string> = yield([?it].extract<interest>().extract<name>());
} \ orderDescendingBy<score>

test sameCompanySharedWeightedInterestClipped(subject:namedObject, var clipAmount:int)
{
	subject -:WORKS_FOR-> company:namedObject <-:WORKS_FOR- person:namedObject;
	iterated it {
		subject -is:INTERESTED_IN-> interest:namedObject <-ip:INTERESTED_IN- person;
	---
		def var sharedWeightedInterest:double = Math::min(is.weight, ip.weight);
	} \ orderDescendingBy<sharedWeightedInterest>
---
	iterated it\orderDescendingBy<sharedWeightedInterest>\keepFirst(clipAmount);
	def var score:double = yield([?it].extract<sharedWeightedInterest>().sum());
	def var interestsString:string = yield([?it].extract<interest>().extract<name>().asString(","));
} \ orderDescendingBy<score>

test fetchByName(var name:string) : (namedObject)
{
	namedObj:namedObject;
	if{ namedObj.name == name; }
	return(namedObj);
}
  \end{grgen}\label{exsimplequeryrules}

\end{example}


\begin{example}\label{exsimplequery}

Let us take a look at some example queries, based on Example~\ref{exsimplequeryrules}, which in turn is based on a test that you can find under \texttt{tests/queries} -- also including some example data.

In the model we introduce a "generic" namedObject used for subjects, i.e. persons, and objects like companies or hobbies, and edges showing employment and interest for subjects (the model shows only a few general types, which is untypical for GrGen, but common for modelling with Neo4j/Cypher where the idea for this example stems from).

We start with some simple queries, asking for the interests of a subject, and the intensity of the interest.
In the rules file, they show up as tests containing the patterns to query for.
Then we continue with some more complex queries, asking for the shared interests of persons working in the same company, and the intensity of the shared interest.
In the rules file, they again show up as tests containing the patterns to query for.

\vspace{3mm}

In order to obtain all interests (\texttt{namedObject}s) all persons are interested in, we drop the following query in \GrShell:

	\begin{grshell}
eval [?interests(null)\keepOneForEach<interestString>]
  \end{grshell}

For an unspecified subject (\texttt{null}, thus for all subjects, due to the \texttt{null} specifier at the parameter in the test, which causes searching), it fetches all interests, in the form of \texttt{array<match<interests>>}. The array is freed from duplicates within the \texttt{interestString} member, so only matches with distinct \texttt{interestString}s remain (the first distinct value from the member in the array is kept).

In order to obtain the overall intensity of interests of a single person, we use the following query in \GrShell:

	\begin{grshell}
eval [?weightedInterests(::subject)].extract<weight>().sum()
  \end{grshell}

From the array of matches is an array of the match members \texttt{weight} of type \texttt{array<double>} extracted (resulting from the equivalently named pattern element), and then reduced to a single scalar value with the array accumulation method \texttt{sum}.

In order to obtain the interests of two persons, we use the following query in \GrShell:

	\begin{grshell}
eval [?interests(::subject)].extract<interestString>().asSet() | [?interests(::anotherSubject)].extract<interestString>().asSet()
  \end{grshell}
	
It fetches the interests of \texttt{::subject} and \texttt{::anotherSubject}, in the form of \texttt{array<match<interests>>}, extracts the \texttt{interestString}s, resulting in two \texttt{array<string>}, transforms them to sets (of type \texttt{set<string>}), and finally joins them with set union \texttt{|}.

\end{example}


\begin{example}
Let us continue Example~\ref{exsimplequery}. In order to obtain the persons working in the same company with the highest amount of shared interests, we drop the following query in \GrShell:

  \begin{grshell}
eval [?sameCompanySharedInterest(::subject)\orderDescendingBy<score>\keepFirst(3)]
  \end{grshell}

It searches for all matches of pattern \texttt{sameCompanySharedInterest} for the given subject, in the form of \texttt{array<match<sameCompanySharedInterest>>}.
The array is then ordered descendingly alongside \texttt{score} (which just gives the amount of shared interests), and clipped to the first 3 matches, thus persons.
Besides the order-by that shows up directly in the query expression, is a kind of group-by employed in the used test, with a common pattern giving the shared part, and the differences being encapsulated in the iterated (we don't obtain a cartesian product of all matches and then group afterwards to factor out common parts, but match common parts and extend them by the differences to give the complete match).
The interesting information from the iterated is extracted and accumulated with an iterated query (cf. \ref{sec:primexpr}), which returns an array of matches of the iterated pattern (\texttt{array<match<sameCompanySharedInterest.it>>}).
(The iterated query can be seen as a get-all-matches rule query inside a rule, and the iterated as a nested all-bracketed rule).

We improve upon the last query by obtaining the persons working in the same company with the highest intensity of interest regarding their shared interests:

  \begin{grshell}
eval [?sameCompanySharedWeightedInterestClipped([?fetchByName("hans")][0].namedObj, ::clipAmount)]
  \end{grshell}
	
It employs a subquery \texttt{fetchByName} to obtain a dedicated person, i.e. \texttt{namedObject} by name, reducing the resulting array of matches first to a single array element (the first one) and then to a single member of the match (using the name from the named graph would be preferable in a model fully fitting to the \GrG-technological space); utilizing a query to compute an argument is also of interest in regular rule applications.
Furthermore, it computes the shared interest in an object as the smaller of the interests of both subjects in the object, accumulating over all objects the subjects are interested in, with a cut-off applied by a \texttt{clipAmount}.
\end{example}

\subsubsection*{Multi Rule Query}

\begin{rail}
  MultiRuleQuery: '[' '?' '[' (Rule + ',') ']' (MatchClassFilterCalls)? MatchClassSpec ']' ;
  MatchClassSpec: backslash '<' 'class' MatchClassIdent '>' ;
\end{rail}\ixnterm{MultiRuleQuery}\ixnterm{MatchClassSpec}

The \emph{MultiRuleQuery} clause applies multiple (distinct) rules or tests, that all implement the same match class, cf. Section~\ref{sec:matchclass}).
It defines a query version of the \emph{MultiRuleExecution}, cf. Section~\ref{sec:multiruleapplication}.
It does not allow return assignments in contrast to the latter one, as no output values are available; only the pattern elements -- but all of them -- are accessible (this holds for all pattern-based query constructs).
In contrast to the \emph{MultiRuleExecution}, a \emph{MatchClassSpec} is required.
It defines the type of the result (it cannot be derived from the rules, as these may implement multiple (shared) match classes).

Note that the multi rule query is a sequence expression -- it cannot be applied in a sequence in the place of a multi rule.

\begin{example}
The rules from Example~\ref{exmatchclass} can be applied as a query with e.g. \\
\verb#eval [?[r,s]\Score.orderDescendingBy<score>\Score.keepFirst(3)\<class Score>]#.
\end{example}


%%%%%%%%%%%%%%%%%%%%%%%%%%%%%%%%%%%%%%%%%%%%%%%%%%%%%%%%%%%%%%%%%%%%%%%%%%%%%%%%%%%%%%%%%%%%%%%%
\section{Query Result Combination (Auto-generated functions)}

You can combine the results of multiple tests or rules with functions that receive arrays of match types as inputs and return an array of match class type.
You may do so for one by manually implemented functions, and for the other by auto-generated functions.

Example \ref{exjoins} shows how to combine the matches of the querying tests \texttt{sameCompany} and \texttt{sharedInterest} in the manner of a SQL natural join statement with the \texttt{naturalJoin} function into a \texttt{SameCompanySharedInterest} match class (cf. Section~\ref{sec:matchclass} for match classes; the natural join is a cartesian product with a selection condition requiring value equivalence for common names (to be seen as columns) in the matches (each one to be seen as a row) of the left array (table) and the right array (table)).

\begin{example}
  \begin{grgen}
test sameCompany(subject:namedObject)
{
	subject -:WORKS_FOR-> company:namedObject <-:WORKS_FOR- person:namedObject;
}

test sharedInterest(subject:namedObject)
{
	subject -:INTERESTED_IN-> interest:namedObject <-:INTERESTED_IN- person:namedObject;
}

match class SameCompanySharedInterest
{
	subject:namedObject;
	company:namedObject;
	person:namedObject;
	interest:namedObject;
}

function naturalJoin(ref matchesSameCompany:array<match<sameCompany>>, ref matchesSharedInterest:array<match<sharedInterest>>) : array<match<class SameCompanySharedInterest>>
{
	def ref res:array<match<class SameCompanySharedInterest>> = array<match<class SameCompanySharedInterest>>[];
	for(matchSameCompany:match<sameCompany> in matchesSameCompany)
	{
		for(matchSharedInterest:match<sharedInterest> in matchesSharedInterest)
		{
			if(matchSameCompany.subject == matchSharedInterest.subject && matchSameCompany.person == matchSharedInterest.person) {
				def ref m:match<class SameCompanySharedInterest> = match<class SameCompanySharedInterest>();
				m.subject = matchSameCompany.subject;
				m.person = matchSameCompany.person;
				m.company = matchSameCompany.company;
				m.interest = matchSharedInterest.interest;
				res.add(m);
			}
		}
	}
	return(res);
}
  \end{grgen}\label{exjoins}
	
	\begin{grgen}
match class SameCompanySharedInterest
{
	auto(match<sameCompany> | match<sharedInterest>)
}

function naturalJoin(ref matchesSameCompany:array<match<sameCompany>>, ref matchesSharedInterest:array<match<sharedInterest>>) : array<match<class SameCompanySharedInterest>>
{
	auto(join<natural>(matchesSameCompany, matchesSharedInterest))
}
  \end{grgen}
\end{example}

The match of the pattern of the test \texttt{sameCompany} contains the named members \texttt{subject}, \texttt{person}, and \texttt{company}, while the match of the pattern of the test \texttt{sharedInterest} contains the named members \texttt{subject}, \texttt{person}, and \texttt{interest}.
They are combined into a \texttt{SameCompanySharedInterest} class, with single \texttt{subject} and \texttt{person} entries (a name can only appear once, you'd have to add a disambiguating prefix in order to keep two entries stemming from fields with same name, but as they'd contain the same entries due to the join anyway, we can just use a single unprefixed entry without loosing information), and \texttt{company} as well as \texttt{interest} members.
The combination is computed from the matches that agree in their \texttt{subject} and \texttt{person} fields (i.e. have equal values), so you get an array of matches of the common companies and shared interests of two persons -- i.e. the persons that work in the same (have one common) company and have a shared (the same) interest, with subject being fixed this amounts to his/her colleagues with shared interests.

The upper part gives a manually implemented version, the lower part shows how to compute the natural join with an auto-generated function body -- with the resulting code being the same as in the manually implemented version.
It also shows how to compute the match class with an auto-generated match class body, combining the specified match types under name merging -- the resulting code is also the same as in the manually implemented version.
The auto-generated match class body is to be used with care, it is more concise and writing it is more convenient than listing the members explicitly, but readability is severly hampered, you have to mentally compute the cross product modulo name merging.
Note that only named pattern entities are contained in the result; the join also ignores (projects away) unnamed entities.

\begin{rail}
  AutoFunctionBody: 'auto' '(' AutoFunctionBodyContent ')';
  AutoFunctionBodyContent: 'join' '<' ('natural' | 'cartesian') '>' '(' VariableIdent ',' VariableIdent ')' | VariableIdent '.' 'keepOneForEach' '<' MemberIdent '>' \\ 'Accumulate' '<' MemberIdent '>' 'By' '<' AccumulationMethod '>';
  AutoMatchClassBody: 'auto' '(' VariableIdent '|' VariableIdent ')';
\end{rail}\ixnterm{AutoFunctionBody}\ixnterm{AutoMatchClassBody}\ixkeyw{auto}

Another auto-generated function body is available for the cartesian product, noted down with \texttt{auto(join<cartesian>(leftArray, rightArray))}.
The implementation is the same like the one for the natural join, with exception of the missing \texttt{if} condition (so no real cartesian join is available in case of common names thus shared fields, you have to prepend/append name suffixes to obtain a result containing the values from both inputs).
The auto generated functions also support arrays of match classes as input.

\subsubsection*{Auto-generated array method}

The equivalent of an SQL GROUP BY statement with an accumulation function can be generated with the \texttt{keepOneForEach<X>Accumulate<Y>By<Z>} filter, cf. Section~\ref{filter:auto-generated}.
It removes the entries in the matches array that are duplicates in/of member X, the member Y of the remaining entry receives the result of accumulating the array of all values of Y of the same X by the array accumulation method Z (the available ones are listed in \ref{arrayaccumulationmethod}).
This \emph{filter} allows to "fuse" results, the quarantine test (to be found under \verb#tests\quarantine#) shows how to use it for summing scores of entries that fall together.

\begin{example}
  \begin{grgen}
match class QuarantineScore
{
	person:Person;
	def var score:double;
} \keepOneForEach<person>Accumulate<score>By<sum>, orderDescendingBy<score>

rule quarantineByRiskLodging implements QuarantineScore
{
	person:Person -:livesIn-> l:Lodging;
	if { l.type == LodgingType::HomeForTheElderly; }
---
	def var score:double = 100.0;
	
	modify {
		person -:underQuarantine-> person;
	}
}
  \end{grgen}\label{exkeeponeforeachaccumulatebyfilter}

	\begin{grshell}
	eval [?[quarantineByRiskLodging, quarantineByRiskHabit, quarantineByRiskIllness]
		\QuarantineScore.keepOneForEach<person>Accumulate<score>By<sum>
		\QuarantineScore.orderDescendingBy<score>\<class QuarantineScore>]
	\end{grshell}
\end{example}

An equivalent array-of-matches processing \emph{function} can be generated with the function body \texttt{auto(target.keepOneForEach<X>Accumulate<Y>By<Z>)}.
(The auto-generation is only supported for arrays of matches but not for arrays of node/edge types, in contrast to the auto-supplying of the array methods listed in \ref{arraybymemberaccessmethod} that support node/edge types (and attributes), because of the side effect of accumulation on the accumulation attribute of the graph element, thus the graph.)
Being a function, it can be applied also after matches array return, in contrast to filters that must be applied directly after matching before the matches array is returned for further processing.
Applying a function is esp. of interest for the use case of combining multiple matches arrays.

\begin{example}
  \begin{grgen}
function keepOneAccumulate(ref matchesSameCompany:array<match<class QuarantineScore>>) : array<match<class QuarantineScore>>
{
	auto(matchesSameCompany.keepOneForEach<person>Accumulate<score>By<sum>)
}
  \end{grgen}\label{exkeeponeforeachaccumulateby}

	\begin{grshell}
	eval keepOneAccumulate([?[quarantineByRiskLodging, quarantineByRiskHabit, quarantineByRiskIllness]\<class QuarantineScore>]).orderDescendingBy<score>()
	\end{grshell}
\end{example}


%%%%%%%%%%%%%%%%%%%%%%%%%%%%%%%%%%%%%%%%%%%%%%%%%%%%%%%%%%%%%%%%%%%%%%%%%%%%%%%%%%%%%%%%%%%%%%%%
\section{Query Looping and Counting Rule Execution}

Instead of querying the graph with a sequence expression, and continuing processing with filters and array accumulation methods, is it possible to query it with a rule in a for loop of the sequences, processing the single matches one-by-one.

\begin{rail}
  ExtendedControl:
    'for' lbrace Variable ':' Type
    ('in' '[' '?' RuleIdent ']' ';')\\
    RewriteSequence rbrace
    ;
\end{rail}\ixkeyw{for}\label{formatch}

The \texttt{for} matches loop allows to iterate through the matches found for an all-bracketed rule reduced to a test (or a test); i.e. the rule is not applied, we only iterate its matches.
The loop variable must be of type \texttt{match<r>} with \texttt{r} being the name of the rule matched (the type must be statically known).
The elements (esp. the nodes and edges) of the pattern of the matched rule can then be accessed by applying the \texttt{.}-operator on the loop variable, giving the name of the element of interest after the dot.
Note: the elements must be assigned to a variable so their attributes can be accessed, a direct attribute access after the match access is not possible.
Note: the match object allows only to access the top level nodes, edges, or variables.
If you use subpatterns or nested patterns and want to access elements found by them, you have to \texttt{yield}(\ref{sub:yield}) them out to the top-level pattern.

In contrast to a rule query that only allows to fetch data from the graph without modifying it, is it possible for a counted rule application to apply a rule (including its rewrite part), and to assign the size of the matches array as rudimentary query result to a variable.

\begin{rail}
  RuleExecution: (() 
	| '(' (Variable+',') ')' '=') \\ ('count' '[' RuleModifier Rule ']' '=''>' Variable);
\end{rail}\ixnterm{RuleExecution}

The \texttt{count}ed all bracketing \texttt{count[r]=>c} assigns the \indexed{count} of matches of rule \texttt{r} to the variable \texttt{c}, and applies \texttt{r} on all the matches.
With \texttt{count[?r]=>c} the matches are only counted, no rewrites are carried out.

% todo: beispiele im text bringen
