\chapter{Advanced Control with Backtracking}\indexmain{advanced control}\label{cha:transaction}

In this chapter we'll have a look at advanced graph rewrite sequence constructs,
with the \indexed{subsequences} and the backtracking double angles as the central statements.


%%%%%%%%%%%%%%%%%%%%%%%%%%%%%%%%%%%%%%%%%%%%%%%%%%%%%%%%%%%%%%%%%%%%%%%%%%%%%%%%%%%%%%%%%%%%%%%%
\section{Sequence Definitions (Procedural Abstraction )} \label{sec:sequencedefinition}
\begin{rail}
  RewriteSequenceDefinition: 
    ('def' | 'sequence') RewriteSequenceSignature lbrace RewriteSequence rbrace;
  RewriteSequenceSignature: 
    SequenceIdentDecl ('(' ((InVariable ':' Type)*',') ')')? \\ (':' '(' ((OutVariable ':' Type)*',') ')')?
	;
\end{rail}\ixnterm{RewriteSequenceDefinition}\ixnterm{RewriteSequenceSignature}

If you want to use a sequence or sequence part at several locations, just factor it out into a \indexed{sequence definition} and reuse with its name as if it were a rule.
A sequence definition declares input and output variables; 
when the sequence gets called the input variables are bound to the values it was called with.
If and only if the sequences succeeds, the values from the output variables get assigned to the assignment target of the sequence call.
Thus a \indexed{sequence call} behaves as a rule call, cf. \ref{sec:ruleapplication}.

A sequence definition may call itself recursively, as can be seen in example \ref{ex:recseq}.

The compiled sequences must start with the \texttt{sequence} keyword in the rule file.
The interpreted sequences in the shell must start with the \texttt{def} keyword; a shell sequence can be overwritten with another shell sequence in case the signature is identical.
(Overwriting is needed in the shell to define directly or mutually recursive sequences, as a sequence must be defined before it can get used; apart from that it allows for a more rapid-prototyping like style of development in the shell.)

\begin{example}
\label{ex:recseq}
\begin{grgen}
def rec(depth:int) {\
  if{ {{depth<::MAXDEPTH}}; foo() ;> rec(depth+1); bar() }\
}
\end{grgen}
This example shows a sequence defined in the shell which is executing itself recursively.
The host graph is transformed by applying \texttt{MAXDEPTH} times the rule \texttt{foo}, until finally the rule \texttt{bar} is executed. The result of the sequence is the result of \texttt{bar}, returned back from recursion step to recursion step while unwinding the sequence call stack. 
\end{example}


%%%%%%%%%%%%%%%%%%%%%%%%%%%%%%%%%%%%%%%%%%%%%%%%%%%%%%%%%%%%%%%%%%%%%%%%%%%%%%%%%%%%%%%%%%%%%%%%
\section{Transactions, Backtracking, and Pause Insertions}\label{sec:extctrl}

The extended control constructs offer further rule application control in the form of \indexed{transaction}s, \indexed{backtracking}, and \indexed{pause insertions}.

\begin{rail} 
  ExtendedControl: 
    '<' RewriteSequence '>' | 
    '<<' RuleExecution ';;' RewriteSequence '>>' |
    '/' RewriteSequence '/'
	;
\end{rail}

Graph rewrite sequences can be processed \indexed{transaction}ally by using angle brackets (\texttt{<>}), i.e.
if the return value of the nested sequence is \texttt{false}, all the changes carried out on the host graph will be rolled back.
Nested transactions\indexmainsee{nested transaction}{transaction} are supported, i.e. a transaction which was committed is rolled back again if an enclosing transaction fails.

\begin{example}
We want to execute a sequence \texttt{rs} and if it fails a rule \texttt{t} -- on the original graph. But \texttt{rs} is composed of two rules, it denotes the sequence \verb#r&&s#. So first \texttt{r} must succeed, and then \texttt{s}, too. Unfortunately, it is possible that \texttt{r} succeeds and carries out its effects on the graph, and then \texttt{s} fails. So we'd need an explicit \texttt{undo-r} to reverse the effect of \texttt{r} after \texttt{s} failed. Luckily, we can omit writing such reversal code (which may require a horrendous amount of bookkeeping) by utilizing transactions, to just try out and roll back on failure: \verb#if{!< r && s >; t}#. After \texttt{r} succeeded -- with the graph in state post-\texttt{r} -- \texttt{s} is executed. If \texttt{s} succeeds, the sequence as such succeeds. But if \texttt{s} fails we just roll back the effects of \texttt{r} and execute \texttt{t} instead. 
\end{example}

Transactions as such are only helpful in a limited number of cases, but they are a key ingredient for backtracking, which is syntactically specified by double angle brackets (\texttt{<<r;;s>>}.
The semantics of the construct are:
First compute all matches for rule \texttt{r}, then start a transaction.
For each match (one-by-one): rewrite the match, then execute \texttt{s}.
If \texttt{s} failed then roll the effects on the graph back and continue with the matches processing loop.
If \texttt{s} succeeded then commit the changes and break from the matches processing loop.
The case one \texttt{s} succeeded, the overall sequence succeeds, otherwise it fails.
On first sight this may not look very impressive, but this construct in combination with recursive sequences is the key operation for crawling through \indexed{search space}s or for the unfolding of \indexed{state space}s.

The backtracking double angles separate matching from rewriting: first all matches are found, but then only one after the other is applied, without interference of the other matches.
The ``without interference of other matches'' statement is ensured by rolling back the changes of the application of the previous match, and much more, of the entire sequence which followed the rewriting of the previous match.

\begin{example}
The situation we find ourselves in now is to find an application of rule \texttt{r} that makes test \texttt{t} succeed. Unfortunately, the test would be very complicated to write based on the state before-\texttt{r}. So we want to execute \texttt{r} and try \texttt{t} afterwards. But remembering the other matches of \texttt{r} we found in case the just chosen match makes \texttt{t} fail, so we can exhaustively enumerate and try them all. Only in case none of them works do we accept failure.
The sequence \verb#<<r;;t>># allows us directly to do so, it fails iff none of the possible applications of \texttt{r} can make a following \texttt{t} succeeds, it succeeds with the first application of \texttt{r} for which \texttt{t} succeeds. 
\end{example}

If you are just interested in the first goal state stumbled upon which satisfies your requirements during a search,
then you only need to give a condition as last statement of the sequence which returns true if the goal was reached; the iteration stops exactly in the target state (see example above).
But when you are interested in finding all states which satisfy your requirements, or even in enumerating each and every state, just force backtracking by noting down the constant false as last element of the sequence (see example below).

\begin{example}
The sequence \verb#<< r ;; t ;> false >># allows us to visit all applications of \texttt{r} that make \texttt{t} succeed.
But everything is rolled back after sequence end.
So this is only helpful if you remember something about the states that saw \texttt{r} and \texttt{t} succeed.
Either in some variables, or in the graph itself, during a transaction pause. 
\end{example}

The backtracking construct encodes a single decision point of a search, splitting into breadth along the different choices available at that point, and further continuing the search in the sequence.
When this point of splitting into breadth is contained in a sequence,
and this sequence calls itself again recursively on each branch of the decision taken,
you get a recursion which is able to search into depth,
continuing decision making on the resulting graph of the previous decision. 

\begin{example}
The example shows the scheme for a backtracking search.
\begin{grgen}
sequence rec(level:int) {
	if{ {{level < ::MAXDEPTH}};
		<< r ;; t && changes-to-be-remembered && rec(level+1) >>;
		false } // end of recursion, will make all the rec return false, causing full rollback to initial state, with exception of changes that occured during a pause or are outside the graph and thus out of reach of transaction control
}
\end{grgen}
\end{example}

\begin{figure}[htbp]
  \centering
  \includegraphics[width=\textwidth]{fig/SearchSpace}
  \caption{Search space illustration, a bullet stands for a graph}
  \label{figsearchspace}
\end{figure}

With each sequence call advancing one step into depth and each backtracking angle advancing into breadth, you receive a depth-first enumeration of an entire search space (as sketched in \ref{figsearchspace}).
Each state is visited in \emph{temporal succession}, with only the most recent state being available in the graph.
But maybe you want to keep each state visited, because you are interested in viewing all results at once, or because you want to compare the different states.
As there is only one host graph in \GrG, keeping each visited state requires a partition of the host graph into separate subgraphs, each denoting a state.

After you changed the modeling from a host graph to a state space graph consisting of multiple subgraphs, each representing one of the graphs you normally work with, 
you can materialize the search space visited in temporal succession into a state space graph,
by copying the subgraphs (which are normally only existing at one point in time) during \indexed{pause insertion}s out into space.
When subgraphs would be copied without pause insertions, they would be rolled back during backtracking; but effects applied on the graph from \texttt{/ in between here /} are bypassing the recording of the transaction undo log and thus stay in the graph, even if the transaction fails and is rolled back. 

When you have switched from a depth-first search over one single current graph to the unfolding of a state space graph containing all the subgraphs reached, you may compare each subgraph which gets enumerated with all the already available subgraphs, and if the new subgraph already exists (i.e. is isomorph to another already generated subgraph), you may refrain from inserting it.
This \indexed{symmetry reduction} allows to save the space and time needed for storing and computing equivalent branches otherwise generated from the equivalent states. 
But please note that the \texttt{==} operator on graphs is optimized for returning early when the graphs are different; when the graphs are isomorphic you have to pay the full price of graph isomorphy checking.
This will happen steadily with \indexed{automorphic pattern}s and then degrade performance.
To counter this filter the matches which cover the same spot in different ways, see \ref{sub:extflt} on how to do this.
Merging states with already computed ones yields a DAG-formed state space, instead of the always tree like search space.
Have a look at the transformation techniques chapter for more on state space enumeration \ref{sec:statespaceenum} and copying \ref{subsub:copystructure}.
One caveat of the transactions and backtracking must be mentioned: rollback might lead to an incorrect graph visualization when employed from the debugger.
This holds especially when using grouping nodes to visualize subgraph containment (\ref{sub:visual}). You must be aware that you can't rely on the online display as much as you can normally, and that you maybe need to fall back to an offline display by opening a \texttt{.vcg}-dump of the graph written in a situation when the online graph looked suspicious; a dump can be written easily in a situation of doubt from the debugger pressing the \texttt{p} key. 

\begin{note}
While a transaction or a backtrack is pending, all changes to the graph are recorded into some kind of undo log, which is used to reverse the effects on the graph in the case of rollback (and is thrown away when the nesting root gets committed).
So these constructs are not horribly inefficient, but they do have their price --- if you need them, use them, but evaluate first if you really do.
\end{note}


%%%%%%%%%%%%%%%%%%%%%%%%%%%%%%%%%%%%%%%%%%%%%%%%%%%%%%%%%%%%%%%%%%%%%%%%%%%%%%%%%%%%%%%%%%%%%%%%
\section{Multi-Rule Backtracking}\label{sec:multitransaction}

\begin{rail}
  MultiBacktracking: '<<' MultiRuleExecution ';;' RewriteSequence '>>';
\end{rail}\ixnterm{MultiBacktracking}

The semantics of the \emph{MultiBacktracking} construct are:
First compute all matches for all rules, then start a transaction.
For each match (one-by-one): rewrite the match, then execute the contained sequence.
If the contained sequence failed then roll the effects on the graph back and continue with the matches processing loop.
If the contained sequence succeeded then commit the changes and break from the matches processing loop.
The case one sequence succeeded, the overall sequence succeeds, otherwise it fails.

Filters may be applied to the rules in the construct (e.g. \verb#r\f#), but especially: match class filters may be applied to the multi rule execution part \verb#([[...]\f])# -- they allow to bring the matches of the different rules into a global order, so only the most promising ones are followed, while the other ones are purged -- the default order follows syntactical appearance, first all matches of the first rule are processed, then all matches of the second rule, and so on.

\begin{rail}
  MultiRulePrefixedSequenceBacktracking: '<<' MultiRulePrefixedSequence '>>';
\end{rail}\ixnterm{MultiRulePrefixedSequenceBacktracking}

The semantics of the \emph{MultiRulePrefixedSequenceBacktracking} construct are:
First compute all matches for all rules, then start a transaction.
For each match (one-by-one): rewrite the match, then execute the corresponding sequence.
If the corresponding sequence failed then roll the effects on the graph back and continue with the matches processing loop.
If the corresponding sequence succeeded then commit the changes and break from the matches processing loop.
The case one sequence succeeded, the overall sequence succeeds, otherwise it fails.

Filters may be applied to the rules in the construct (e.g. \verb#r\f#), but especially: match class filters may be applied to the multi rule prefixed sequence part \verb#([[...]\f])# -- they allow to bring the matches of the different rules into a global order, so only the most promising ones are followed, while the other ones are purged -- the default order follows syntactical appearance, first all matches of the first rule are processed, then all matches of the second rule, and so on.

%%%%%%%%%%%%%%%%%%%%%%%%%%%%%%%%%%%%%%%%%%%%%%%%%%%%%%%%%%%%%%%%%%%%%%%%%%%%%%%%%%%%%%%%%%%%%%%%
\section{For Loops}

\begin{rail}
  ExtendedControl:
    'for' lbrace Variable ':' Type\\
    ('in' Function '(' Parameters ')' ';' |
		'in' '[' SequenceExpression ':' SequenceExpression ']' ';' |
		'in' IndexAccess ';')\\
    RewriteSequence rbrace
    ;
\end{rail}\ixkeyw{for}\label{forgraphelem}\label{forincidentadjacent}

The \texttt{for} loop over the \emph{Functions} \texttt{nodes} or \texttt{edges} are iterating over all the elements in the current host graph which are compatible to the type given.
The iteration variable is bound to the currently enumerated graph element, then the sequence in the body is executed.

If you iterate a node type from a graph, you may be interested in iterating its incident edges or its adjacent nodes.
This can be achieved with a for neighbouring elements loop, which binds the iteration variable to an edge in case the \emph{Function} is one of \texttt{incoming}, \texttt{outgoing}, or \texttt{incident}. 
Or which binds the iteration variable to a node in case the \emph{Function} is one of \texttt{adjacentIncoming}, \texttt{adjacentOutgoing}, or \texttt{adjacent}.

Moreover, you may iterate with \texttt{reachable\-Incoming}, \texttt{reachable\-Outgoing}, and \texttt{reachable} the nodes reachable from a starting node, or with \texttt{reachable\-Edges\-Incoming}, \texttt{reachable\-Edges\-Outgoing}, and \texttt{reachable\-Edges} the edges reachable from a starting node.
Or you may use one of the bounded reachability functions \texttt{bounded\-Reachable}, \texttt{bounded\-Reachable\-Incoming}, \texttt{bounded\-Reachable\-Outgoing}, \texttt{bounded\-Reachable\-Edges}, \texttt{bounded\-Reachable\-Edges\-Incoming}, \texttt{bounded\-Reachable\-Edges\-Outgoing} here.

The admissible \emph{Parameters} for the \emph{Functions} are the source node, or the source node plus the incident edge type, or the source node plus the incident edge type, plus the adjacent node type ---
that's the same as for the sequence expression functions explained in \ref{neighbouringelementsfunctions}/Connectedness queries.
In contrast to these set returning functions, this loop contained functions enumerate nodes/edges multiple times in case of reflexive or multi edges.

The third \texttt{for} loop from the diagram above, the for integer range loop, allows to cycle through an ascending or descending series of integers; the loop variable must be of type \texttt{int}.
First the left and right sequence expressions are evaluated,
if the left is lower or equal than the right the variable is incremented from left on in steps of one until right is reached (inclusive),
otherwise the variable is decremented from left on in steps of one until right is reached (inclusive).

The fourth \texttt{for} loop listed above allows to iterate the contents of an index, see \ref{sub:indexusage} for more on this and the \emph{IndexAccess}.

The most important \texttt{for} loop, the one iterating a container, for enumerating the elements contained in storages, was already introduced here: \ref{forstorage}.
All \texttt{for} loops fail if one of the sequence executions from the body failed (all are carried out, though, even after a failure), and succeed otherwise.


%%%%%%%%%%%%%%%%%%%%%%%%%%%%%%%%%%%%%%%%%%%%%%%%%%%%%%%%%%%%%%%%%%%%%%%%%%%%%%%%%%%%%%%%%%%%%%%%
\section{Mapping Clause} \label{sec:mappingclause}

%TODO: in general graph mapping compared to graph rewriting, creating new graphs, side-effect-free
%\chapter{Graph Mapping}\indexmain{graph mapping}\label{cha:graphmapping}

\begin{rail}
  MappingClause: '[' ':' ((RulePrefixedSequenceAtom) + (',')) (MatchClassFilterCalls)? ':' ']' ;
\end{rail}\ixnterm{MappingClause}

The \emph{MappingClause} yields an array of new graphs derived from the input host graph.
Note that the host graph is \emph{not} changed by this sequence expression.

The example construct \verb#[:for{r;seq}:]# returns the graphs for which executing \texttt{seq} yielded true, after applying a match of \texttt{r}, on a clone of the current host graph, for all matches found.
It can be extended to multiple rules and sequences \verb#[:for{r1;seq1},for{r2;seq2}:]#, you receive then the array concatenation of the single constructs.

The \emph{MappingClause} should be handled with care, as for each match, a clone of the current host graph is created, which is expensive.
So it should only be used if this is not of major importance, or what is expected anyway, as e.g. in a state space enumeration.
A simple example implementation of such one can be found in the \texttt{tests/statespaceMapping} folder.
In a typical full implementation, you would utilize a tree of object class instances (cf. Section~\ref{sec:objecttypes}), storing the graphs reached during unfolding (or a DAG of object class instances, in case isomorphic graph pruning is carried out).


%%%%%%%%%%%%%%%%%%%%%%%%%%%%%%%%%%%%%%%%%%%%%%%%%%%%%%%%%%%%%%%%%%%%%%%%%%%%%%%%%%%%%%%%%%%%%%%%
\section{Quick Reference Table}

Table~\ref{seqtab} lists most of the operations of the advanced graph rewrite sequence constructs at a glance, followed by the language constructs supporting graph nesting.

%\makeatletter
\begin{table}[htbp]
\begin{minipage}{\linewidth} \renewcommand{\footnoterule}{} 
\begin{tabularx}{\linewidth}{|lX|}
\hline
\texttt{<s>} & Execute \texttt{s} transactionally (rollback on failure).\\
\texttt{<<r;;s>>} & Backtracking: try the matches of \texttt{r} until \texttt{s} succeeds.\\
\texttt{/ s /} & Pause insertion: execute \texttt{s} outside of the enclosing transactions and sequences, i.e. the changes of \texttt{s} are not rolled back.\\
\hline
\texttt{<<[[r1,r2]];;s>>} & Multi-Rule-Backtracking: try the matches of \texttt{r1} and \texttt{r2} until \texttt{s} succeeds. You may apply a match class filter \texttt{\textbackslash mc.f}, the matches are then tried in the order left behind by the match class filter.\\
\texttt{<<[[for\{r1;s\},for\{r2;t\}]]>>} & Multi-Rule-Prefixed-Sequence-Backtracking: try the matches of \texttt{r1} and \texttt{r2} until \texttt{s} or \texttt{t} succeeds. You may apply a match class filter \texttt{\textbackslash mc.f}, the matches are then tried in the order left behind by the match class filter.\\
\hline
\texttt{for\{v in u; s\}} & Execute \texttt{s} for every \texttt{v} in storage set \texttt{u}. One \texttt{s} failing pins the execution result to failure.\\
\texttt{for\{v->w in u; s\}} & Execute \texttt{s} for every pair (\texttt{v},\texttt{w} in storage map \texttt{u}. One \texttt{s} failing pins the execution result to failure.\\
\texttt{for\{v in func(...); s\}} & Execute \texttt{s} for every node/edge in func, which may be nodes/edges, or incident/adjacent, or reachable/boundedReachable. One \texttt{s} failing pins the execution result to failure.\\
\texttt{for\{v:int in [l:u]; s\}} & Execute \texttt{s} for every \texttt{v} in the integer range starting at \texttt{l} and ending at \texttt{u}, upwards by one if \texttt{l<=r}, otherwise downwards by one. One \texttt{s} failing pins the execution result to failure.\\
\texttt{for\{v:N in \{asc.(idx>7)\}; s\}} & Execute \texttt{s} for every \texttt{v} in the index \texttt{idx}, in ascending order, from \texttt{7} exclusive on. One \texttt{s} failing pins the execution result to failure. \texttt{asc.} abbreviates \texttt{ascending}, is not valid syntax as such.\\
\hline
\texttt{\{comp\}}	& An unspecified sequence computation (see table \ref{comptab}).\\
\hline
\texttt{(w)=s(w)} & Calls a sequence \texttt{s} handing in \texttt{w} as input and writing its output to \texttt{w}; defined e.g. with \texttt{sequence s(u:Node):(v:Node)} \texttt{\{ v=u \}}.\\
\hline
\hline
\texttt{[var] g:graph}	& Declares a variable \texttt{g} of type graph, as attribute inside a node or edge class, or as local variable or parameter.\\
\hline
\texttt{in g \{s\}}	& Executes \texttt{s} in the graph \texttt{g}.\\
\hline
\texttt{(...)=g.r(...)} & Executes the rule \texttt{r} in the graph \texttt{g}\\
\hline
\texttt{this} & Returns the current graph, can be used like a constant variable in an expression, of the rule language, or the sequence computations. Inside a method, \texttt{this} denotes the object of the containing node or edge!\\
\hline
\end{tabularx}\indexmain{\texttt{<>}}\indexmain{\texttt{<<;>>}}
\end{minipage}\\
\\ 
{\small Let \texttt{r}, \texttt{r1}, \texttt{r2} be rules or tests, \texttt{s}, \texttt{t} be sequences, \texttt{u}, \texttt{v}, \texttt{w} be variable identifiers, \texttt{g} be a variable or attribute of graph type, \texttt{<op>} be $\in \{\texttt{|}, \texttt{\textasciicircum}, \texttt{\&}, \texttt{||}, \texttt{\&\&}\}$ }%and \texttt{n}, \texttt{m} $\in \N_0$.}
\caption{Advanced sequences and graph nesting support at a glance}
\label{seqtab}
\end{table}
%\makeatother
 
% todo: beispiele im text bringen
