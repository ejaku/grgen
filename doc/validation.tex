\chapter{Validation and Inspection}\indexmain{GrShell}
\label{chapvalidation}

In order to validate, i.e. test your graph rewrite system, be it manually or automatically, you may employ the means described in this chapter (note that due to its well-founded semantics, even formal verification would be possible, cf. Chapter~\ref{chp:designgoals}).

Available are model and graph queries, they allow you to fetch information of interest from the (meta) model and/or the host graph (instance), be it for validation or just for inspection.
Also available are dedicated validation commands, they allow you to check the connection assertions of the model or that a condition expressed as a sequence holds (in contrast to a normal sequence execution that may be used for the same purpose, the sequence may carry out destructive changes, as it operates on a clone of the graph).

When you use \GrG{} at API level, you are free to employ a .NET test framework of your choice.
Alternatively, you may use the \GrShell{} for testing.
We use the \GrShell{} together with bash shell scripts for automated testing of the integrity of \GrG{} itself.
You may employ the same scripts for testing your project (if you use Windows, you need the cygwin environment).
The testing workflow consists of the application of two scripts:
\begin{itemize}
	\item the bash shell script \texttt{gentest.sh} records the textual output of certain \GrShell{}-commands.
	\item the bash shell script \texttt{test.sh} compares the recorded output with the current output of the \GrShell{} and reports differences as test failures.
\end{itemize}
You find them in the \texttt{tests} folder (of \texttt{engine-net-2} when you are working with the sources --- the \texttt{test} folder (from the \texttt{frontend} directory when you are working with the sources) contains further scripts for compilation-only tests --- both test suites are quite slow in execution, as they are not unit tests).

Besides the \texttt{show} and \texttt{validate} commands defined in this chapter in the following sections, also the result of the \texttt{show var} command (cf. \ref{secshellvariables}) is recorded (and checked against).
In addition, the results of sequence \texttt{exec}ution (a boolean value) and sequence expression \texttt{eval}uation (cf. \ref{grsthings}) are recorded (and checked against),
the same holds for the number of \texttt{matches found} and \texttt{rewrites performed} during execution of a sequence (expression).
(Even if this is a bit ugly: 
\verb#emit("the value of variable \"x\" of type dontcare is: ", x, "\n")#
--- by emitting from your rules or procedures to stdout in a way like the commands do, you could add values to check to a test, the statement just introduced would allow to check that the value of the variable \texttt{x} stays the same once the testdatafile was generated (an alternative would be to adapt the test scripts).)

You may also use the return code of the \GrShell{} for testing when it is started with the \texttt{-N} parameter,
in normal interactive mode it will abort script file execution when an error occurs so the user can continue at this point, in non-interactive mode asked for with the parameter, instead an error code is returned when an error occurs or a validation fails, see Chapter~\ref{cha:toolsandcomponents} for more on this.


%%%%%%%%%%%%%%%%%%%%%%%%%%%%%%%%%%%%%%%%%%%%%%%%%%%%%%%%%%%%%%%%%%%%%%%%%%%%%%%%%%%%%%%%%%%%%%%%
\section{Model and Graph Queries}

We start with the queries returning information about the static types (from the (meta) model).

\begin{rail}
  'show' EntityKind 'types'
\end{rail}\ixkeyw{show}\ixkeyw{types}
Prints the types of the respective entity kind (\texttt{node}/\texttt{edge}/internal class \texttt{object}/\texttt{transient} internal class \texttt{object}) from the current graph model.

\begin{rail}
  'show' EntityKind ('super' | 'sub') 'types' EntityType
\end{rail}\ixkeyw{show}\ixkeyw{super}\ixkeyw{sub}\ixkeyw{types}\indexmain{inheritance}
Prints the inherited/descendant types of the given \emph{Entity\-Type} (\emph{Node\-Type}/\emph{Edge\-Type}/\emph{Object\-Type}/\emph{Transient\-Object\-Type}); the kind identifier must fit to the specified type.

\begin{rail}
  'show' EntityKind 'attributes' (() | (() | 'only') EntityType)
\end{rail}\ixkeyw{show}\ixkeyw{only}\ixkeyw{attributes}
Prints the available \indexed{attribute} types.
If an \emph{EntityType} is supplied, only attributes defined in the /\emph{EntityType} are diplayed (otherwise all attributes known to all all types of the specified \emph{EntityKind}).
The \texttt{only} keyword excludes inherited attributes.
%\begin{warning}
%The \texttt{show EntityKind attributes\dots} command covers types and \emph{inherited} types.
%This is in contrast to the other \texttt{show\dots} commands where types and \emph{sub}types are specified or the direction in the type hierarchy is specified explicitly, respectively.
%\end{warning}

Now to the queries returning information about the dynamic values from the current graph.

\begin{rail}
  'show' (() | 'num') ('nodes' (() | (() | 'only') NodeType) | 'edges' (() | (() | 'only') EdgeType))
\end{rail}\ixkeyw{show}\ixkeyw{num}\ixkeyw{nodes}\ixkeyw{edges}\ixkeyw{only}
Prints the \texttt{num}ber of nodes/edges in the graph, or their actual instances in form of their \indexed{persistent name} together with their type.
If a node type or edge type is supplied, only elements compatible to this type are shown/considered.
The \texttt{only} keyword excludes subtypes.

\begin{rail}
 'show' EntityKind Entity
\end{rail}\ixkeyw{show}
Prints the attribute types and values of a specific entity (again, the \emph{EntityKind} must match the \emph{Entity}).

\begin{rail}
  'show' Entity '.' AttributeName
\end{rail}\ixkeyw{show}
Displays the value of the specified attribute.

\begin{rail}
  EntityKind 'type' Entity 'is' Entity
\end{rail}\ixkeyw{type}\ixkeyw{is}
Prints the information whether the first entity is \indexed{type-compatible}\indexmainsee{compatible types}{type-compatible} to the second entity (the entities must be of the same kind as specified with the \emph{EntityKind}).


%%%%%%%%%%%%%%%%%%%%%%%%%%%%%%%%%%%%%%%%%%%%%%%%%%%%%%%%%%%%%%%%%%%%%%%%%%%%%%%%%%%%%%%%%%%%%%%%
\section{Validation Commands}\label{sec:validate}

\GrG\ offers two different graph validation mechanisms,
the first checks against the connection assertions specified in the model,
the second checks against an arbitrary graph rewrite sequence containing arbitrary tests and rules.

\begin{rail}
  'validate' ('exitonfailure')? ('strict' ('only' 'specified')?)?
\end{rail}\ixkeyw{validate}\ixkeyw{exitonfailure}\ixkeyw{strict}\ixkeyw{only}\ixkeyw{specified}
Validates\indexmain{validate} if the current working graph fulfills the \indexed{connection assertion}s specified in the corresponding graph model (cf. ~\ref{sct:ConnectionAssertions}).
Validate without the strict modifier checks the multiplicities of the connections it finds in the host graph,
it ignores node-edge-node connections which are available in the host graph but have not been specified in the model.
The \emph{strict} mode additionally requires that all the edges available in the host graph must have been specified in the model.
This requirement is too harsh for models where only certain parts are considered critical enough to be checked
or might be a too big step in tightening the level of structural checking in an already existing large model.
So some form of selective strict checking is supported:
The \emph{strict only specified} mode requires strict matching (i.e. that all edges are covered) only of the edges for which connection assertions have been specified in the model.

\begin{rail}
  'validate' ('exitonfailure')? ('exec'|'xgrs') GRS
\end{rail}\ixkeyw{validate}\ixkeyw{exitonfailure}\ixkeyw{xgrs}\ixkeyw{exec}
Validates\indexmain{validate} if the current working graph satisfies the \indexed{graph rewrite sequence} given.
Before the graph rewrite sequence is executed, the instance graph gets cloned;
the sequence operates on the clone, allowing you to change the graph as you want to, without influence on the host graph.
Validation fails iff the sequence fails.
This gives a rather costly but extremely flexible and powerful mechanism to specify graph constraints.
The GrShell is exited with an error code if \texttt{exitonfailure} is specified and the validation fails.

\begin{example}
We reuse a simplified version of the road map model from Chapter~\ref{chapmodellang}:
\begin{grgen}
model Map;

node class city;
node class metropolis;

edge class street;
edge class highway
      connect metropolis [+] --> metropolis [+];
\end{grgen}
The node constraint on \emph{highway} requires all the metropolises to be connected by highways. Now have a look at the following graph:
\begin{center}
  \fbox{\includegraphics[width=8.5cm]{fig/map}}
\end{center}

This graph is valid but not strict valid.
\begin{grshell}
> validate
The graph is valid.
> validate strict only specified
The graph is NOT valid:
  CAE: city "Eppstein" -- highway "A3" --> metropolis "Frankfurt" not specified
> validate strict
The graph is NOT valid:
  CAE: city "Eppstein" -- highway "A3" --> metropolis "Frankfurt" not specified
  CAE: metropolis "Karlsruhe" -- street "trail" --> metropolis "Frankfurt" not specified
>
\end{grshell}
\end{example}

\pagebreak

%%%%%%%%%%%%%%%%%%%%%%%%%%%%%%%%%%%%%%%%%%%%%%%%%%%%%%%%%%%%%%%%%%%%%%%%%%%%%%%%%%%%%%%%%%%%%%%%
\section{Queries with Graphical Output}

The following queries show information graphically (and thus are not amenable to test automatization, or at least less well suited), with the currently configured graph viewer/debugger.

\begin{rail}
  'gshow' 'model'
\end{rail}\ixkeyw{gshow}\ixkeyw{model}
Shows the graph model as a graph similar to an UML inheritance diagram.

\begin{rail}
  'gshow' 'patterns'
\end{rail}\ixkeyw{gshow}\ixkeyw{patterns}
Shows the patterns of the actions as a graph (note that edges are rendered as nodes, so that pattern nesting can be rendered correctly).

