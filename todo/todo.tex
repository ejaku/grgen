\NeedsTeXFormat{LaTeX2e}
\documentclass[12pt,a4paper]{article}
\usepackage{german,a4,latexsym,graphicx,amssymb,color,amsmath}




\begin{document}



\section*{\LARGE The GrGen.NET ToDo-List}



\subsection*{Frontend (Java):}
ToDo:
\begin{itemize}
	\item Invent an attribute type {\tt object} that can only assigned and \emph{nothing} else (maybe we allow test for equality and inequality, i.e. {\tt ==} and {\tt !=}, but that's all).
	\item Check the cause, why some test cases still do not work as required.
		Dont forget to check should\_fail for cases that do \emph{crash} instead of reporting an error.
	\item Improve error messages: Some of them are errorneous and some are even completely malformed. Possible reason: implemented too generic. Possible Location in Code: Resolvers. Possible solution: Implement some more resolvers (but only for error reporting reasons).
    \item {\tt test} rules don't work yet.
	\item For empty .grg-files and for .grg-files that are empty except for a {\tt using} statement GrGen wrongly reports an error.\\
		QUESTION: What is the desired output for such an empty file?
	\item Where warnings should be raised:
		\begin{itemize}
			\item In modify parts, if a graph element occurs \emph{inside} as well as \emph{outside} a {\tt delete} statement.
			\item If an assignment {\tt x.a = \dots} occurs inside an {\tt eval}, but the graph element {\tt x} will be deleted on a rewrite.
			\item If one or more attributes of a newly created node are not initialized by the eval part of the respective rule.
			\item If the types of potentially homomorphic pattern nodes have no common subtype (because non-injective matching is \emph{impossible} in this case).
		\end{itemize}
	\item In some error messages appear corrupt coordinates (this is where a "`?"' appears instread of line and column).
\end{itemize}

\noindent
Done:
\begin{itemize}
	\item Where warnings should be raised:
		\begin{itemize}
			\item On the RHS, if for a returned element homomorphic matching is allowed with a deleted node (hom-delete-return conflict).\\
				\{implemented---Batz 1. Aug. 2007\}
		\end{itemize}
    \item GrGen crashes with a NullPointerException if the .grg file specifies a non-existing model file.\\
		\{fixed---Batz 30. Jul. 2007\}
	\item Implement enums correctly. This includes implicit type casts to {\tt int} as well as use of already defined items in an enum delcaration as well as use of foreign enum items if they are \emph{fully} qualified.
		If they are \emph{not} qualified an error must be reported.
		The def-before-use law must also hold if items from other enum types are used in the definition of an enum item.\\
		\{Seems to work now---Batz 30. Jul. 2007\}
	\item The statements {\tt actions} and {\tt model} should be removed from the specification language of GrGen.NET.
		However, if such a statement occurs at the beginning of a specification, a deprecated-warning should be raised.
		The {\tt using} keyword will be kept, of course.\\
		\{I did it---Batz 25. Jul. 2007\}
	\item In declarations of enumeration types (keyword {\tt enum}) user defined integer values can be assigned to elements.
	  However, on the RHS of such "`assignments"' it should be possible to use already defined elements of that enumeration types in expressions (e.g., \dots{}{\tt{}x = 42, y = x + 3}\dots). However, GrGen does not accept this.\\
		\{works---Batz 24. Jul. 2007\}
	\item Bug: If an error is reported in a model (.gm) file, the execution of the frontend is not aborted before building the IR.\\
		\{fixed---Batz 24. Jul. 2007\}
	\item In replace/modify-part {\tt typeof} does not work when used with retyping of nodes and/or edges.\\
		\{seems to work now---Batz 18. Jul. 2007\}
    \item GrGen crashes when a newly created edge is returned, saying that ''the element 'type3', that isneither a parameter, nor contained in LHS, nor in RHS, occurs in a return''.
		Crashes in\\
		...ast.RuleDeclNode.checkReturnedElemsNotDeleted.\\
		(should\_pass/ret\_003\_fe.grg)\\
		\{fixed---Batz 17. Jul. 2007\}
	\item In modify-parts the error detection wrongly reports invalid reuse of nodes and edges.\\
		\{seems to work now---Batz 17. Jul. 2007\}
	\item Dangling edge graphlets on the RHS should work, if the edge is a reused one and if all incident pattern nodes of that edge are also reused.
    However, if such an edge is retyped, GrGen wrongly reports an error.\\
		\{seems to work now---Batz 17. Jul. 2007\}
	\item Annotions of anonymous nodes and edges do not work.\\
		\{implemented---Batz 14. Jul. 2007\}
	\item If the filename of a .grg-file does not conform with the name given along with the {\tt actions} keyword, an error is raised (which is just the right behaviour). However, the output file is genrated all the same, which should not happen.\\
    \{fixed---Batz 13. Jul. 2007\}
	\item Error detection for the {\tt return} statement is errorneous.\\
		\{seems to work now---Batz 13. Jul. 2007\}
	\item At test case should\_fail/ret\_001.grg:
  	The signature of the rule demands a type {\tt AB}.
	However, if you return a type {\tt C} (that is no subtype of {\tt AB}) no error is reported, which were the right behaviour.\\
		\{as the return stuff now works, this works, too---Batz 13. Jul. 2007\}
\end{itemize}



\subsection*{C\#-Searchplan-Backend (Java):}
ToDo:
\begin{itemize}
	\item Check wether the code generated for access to enum type attributes and constants is correct.
	\item The C\#-code generated for {/should\_pass/basic\_027.grg} is not correct. Because of ''x -e-$>$ y;'' in the modify part, ''delete(x);'' is ignored.
\end{itemize}
Done:
\begin{itemize}
    \item Enum constants are printed as integers in enum expressions.\\
    \{fixed---Kroll 19. Jul. 2007\}
    \item Attributes with names of reserved keywords should be prefixed by an '@'.\\
    \{fixed---Kroll 19. Jul. 2007\}
    \item Fixed casts to strings.\\
    \{fixed---Kroll 19. Jul. 2007\}
    \item Fixed conditions containing casts.\\
    \{fixed---Kroll 19. Jul. 2007\}
    \item Float constants are not correctly emitted.\\
    \{fixed---Kroll 17. Jul. 2007\}
\end{itemize}


\subsection*{Backend (C\#):}
ToDo:
Done:



\subsection*{Other things like, e.g., bugs of unknown origin}
ToDo:
Done:


\end{document}
